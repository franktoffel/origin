
% Default to the notebook output style

% Documento generado automaticamente mediante pandoc
% ipython nbconvert --to latex originlab-python.ipynb    

% Para obtener PDF sustituir $\LaTeX$ por \LaTeX
% Eliminar animación gif


% Inherit from the specified cell style.




    
\documentclass{article}

    
    
    \usepackage{graphicx} % Used to insert images
    \usepackage{adjustbox} % Used to constrain images to a maximum size 
    \usepackage{color} % Allow colors to be defined
    \usepackage{enumerate} % Needed for markdown enumerations to work
    \usepackage{geometry} % Used to adjust the document margins
    \usepackage{amsmath} % Equations
    \usepackage{amssymb} % Equations
    \usepackage[mathletters]{ucs} % Extended unicode (utf-8) support
    \usepackage[utf8x]{inputenc} % Allow utf-8 characters in the tex document
    \usepackage{fancyvrb} % verbatim replacement that allows latex
    \usepackage{grffile} % extends the file name processing of package graphics 
                         % to support a larger range 
    % The hyperref package gives us a pdf with properly built
    % internal navigation ('pdf bookmarks' for the table of contents,
    % internal cross-reference links, web links for URLs, etc.)
    \usepackage{hyperref}
    \usepackage{longtable} % longtable support required by pandoc >1.10
    

    
    
    \definecolor{orange}{cmyk}{0,0.4,0.8,0.2}
    \definecolor{darkorange}{rgb}{.71,0.21,0.01}
    \definecolor{darkgreen}{rgb}{.12,.54,.11}
    \definecolor{myteal}{rgb}{.26, .44, .56}
    \definecolor{gray}{gray}{0.45}
    \definecolor{lightgray}{gray}{.95}
    \definecolor{mediumgray}{gray}{.8}
    \definecolor{inputbackground}{rgb}{.95, .95, .85}
    \definecolor{outputbackground}{rgb}{.95, .95, .95}
    \definecolor{traceback}{rgb}{1, .95, .95}
    % ansi colors
    \definecolor{red}{rgb}{.6,0,0}
    \definecolor{green}{rgb}{0,.65,0}
    \definecolor{brown}{rgb}{0.6,0.6,0}
    \definecolor{blue}{rgb}{0,.145,.698}
    \definecolor{purple}{rgb}{.698,.145,.698}
    \definecolor{cyan}{rgb}{0,.698,.698}
    \definecolor{lightgray}{gray}{0.5}
    
    % bright ansi colors
    \definecolor{darkgray}{gray}{0.25}
    \definecolor{lightred}{rgb}{1.0,0.39,0.28}
    \definecolor{lightgreen}{rgb}{0.48,0.99,0.0}
    \definecolor{lightblue}{rgb}{0.53,0.81,0.92}
    \definecolor{lightpurple}{rgb}{0.87,0.63,0.87}
    \definecolor{lightcyan}{rgb}{0.5,1.0,0.83}
    
    % commands and environments needed by pandoc snippets
    % extracted from the output of `pandoc -s`
    \DefineVerbatimEnvironment{Highlighting}{Verbatim}{commandchars=\\\{\}}
    % Add ',fontsize=\small' for more characters per line
    \newenvironment{Shaded}{}{}
    \newcommand{\KeywordTok}[1]{\textcolor[rgb]{0.00,0.44,0.13}{\textbf{{#1}}}}
    \newcommand{\DataTypeTok}[1]{\textcolor[rgb]{0.56,0.13,0.00}{{#1}}}
    \newcommand{\DecValTok}[1]{\textcolor[rgb]{0.25,0.63,0.44}{{#1}}}
    \newcommand{\BaseNTok}[1]{\textcolor[rgb]{0.25,0.63,0.44}{{#1}}}
    \newcommand{\FloatTok}[1]{\textcolor[rgb]{0.25,0.63,0.44}{{#1}}}
    \newcommand{\CharTok}[1]{\textcolor[rgb]{0.25,0.44,0.63}{{#1}}}
    \newcommand{\StringTok}[1]{\textcolor[rgb]{0.25,0.44,0.63}{{#1}}}
    \newcommand{\CommentTok}[1]{\textcolor[rgb]{0.38,0.63,0.69}{\textit{{#1}}}}
    \newcommand{\OtherTok}[1]{\textcolor[rgb]{0.00,0.44,0.13}{{#1}}}
    \newcommand{\AlertTok}[1]{\textcolor[rgb]{1.00,0.00,0.00}{\textbf{{#1}}}}
    \newcommand{\FunctionTok}[1]{\textcolor[rgb]{0.02,0.16,0.49}{{#1}}}
    \newcommand{\RegionMarkerTok}[1]{{#1}}
    \newcommand{\ErrorTok}[1]{\textcolor[rgb]{1.00,0.00,0.00}{\textbf{{#1}}}}
    \newcommand{\NormalTok}[1]{{#1}}
    
    % Define a nice break command that doesn't care if a line doesn't already
    % exist.
    \def\br{\hspace*{\fill} \\* }
    % Math Jax compatability definitions
    \def\gt{>}
    \def\lt{<}
    % Document parameters
    \title{originlab-python}
    
    
    

    % Pygments definitions
    
\makeatletter
\def\PY@reset{\let\PY@it=\relax \let\PY@bf=\relax%
    \let\PY@ul=\relax \let\PY@tc=\relax%
    \let\PY@bc=\relax \let\PY@ff=\relax}
\def\PY@tok#1{\csname PY@tok@#1\endcsname}
\def\PY@toks#1+{\ifx\relax#1\empty\else%
    \PY@tok{#1}\expandafter\PY@toks\fi}
\def\PY@do#1{\PY@bc{\PY@tc{\PY@ul{%
    \PY@it{\PY@bf{\PY@ff{#1}}}}}}}
\def\PY#1#2{\PY@reset\PY@toks#1+\relax+\PY@do{#2}}

\expandafter\def\csname PY@tok@gd\endcsname{\def\PY@tc##1{\textcolor[rgb]{0.63,0.00,0.00}{##1}}}
\expandafter\def\csname PY@tok@gu\endcsname{\let\PY@bf=\textbf\def\PY@tc##1{\textcolor[rgb]{0.50,0.00,0.50}{##1}}}
\expandafter\def\csname PY@tok@gt\endcsname{\def\PY@tc##1{\textcolor[rgb]{0.00,0.27,0.87}{##1}}}
\expandafter\def\csname PY@tok@gs\endcsname{\let\PY@bf=\textbf}
\expandafter\def\csname PY@tok@gr\endcsname{\def\PY@tc##1{\textcolor[rgb]{1.00,0.00,0.00}{##1}}}
\expandafter\def\csname PY@tok@cm\endcsname{\let\PY@it=\textit\def\PY@tc##1{\textcolor[rgb]{0.25,0.50,0.50}{##1}}}
\expandafter\def\csname PY@tok@vg\endcsname{\def\PY@tc##1{\textcolor[rgb]{0.10,0.09,0.49}{##1}}}
\expandafter\def\csname PY@tok@m\endcsname{\def\PY@tc##1{\textcolor[rgb]{0.40,0.40,0.40}{##1}}}
\expandafter\def\csname PY@tok@mh\endcsname{\def\PY@tc##1{\textcolor[rgb]{0.40,0.40,0.40}{##1}}}
\expandafter\def\csname PY@tok@go\endcsname{\def\PY@tc##1{\textcolor[rgb]{0.53,0.53,0.53}{##1}}}
\expandafter\def\csname PY@tok@ge\endcsname{\let\PY@it=\textit}
\expandafter\def\csname PY@tok@vc\endcsname{\def\PY@tc##1{\textcolor[rgb]{0.10,0.09,0.49}{##1}}}
\expandafter\def\csname PY@tok@il\endcsname{\def\PY@tc##1{\textcolor[rgb]{0.40,0.40,0.40}{##1}}}
\expandafter\def\csname PY@tok@cs\endcsname{\let\PY@it=\textit\def\PY@tc##1{\textcolor[rgb]{0.25,0.50,0.50}{##1}}}
\expandafter\def\csname PY@tok@cp\endcsname{\def\PY@tc##1{\textcolor[rgb]{0.74,0.48,0.00}{##1}}}
\expandafter\def\csname PY@tok@gi\endcsname{\def\PY@tc##1{\textcolor[rgb]{0.00,0.63,0.00}{##1}}}
\expandafter\def\csname PY@tok@gh\endcsname{\let\PY@bf=\textbf\def\PY@tc##1{\textcolor[rgb]{0.00,0.00,0.50}{##1}}}
\expandafter\def\csname PY@tok@ni\endcsname{\let\PY@bf=\textbf\def\PY@tc##1{\textcolor[rgb]{0.60,0.60,0.60}{##1}}}
\expandafter\def\csname PY@tok@nl\endcsname{\def\PY@tc##1{\textcolor[rgb]{0.63,0.63,0.00}{##1}}}
\expandafter\def\csname PY@tok@nn\endcsname{\let\PY@bf=\textbf\def\PY@tc##1{\textcolor[rgb]{0.00,0.00,1.00}{##1}}}
\expandafter\def\csname PY@tok@no\endcsname{\def\PY@tc##1{\textcolor[rgb]{0.53,0.00,0.00}{##1}}}
\expandafter\def\csname PY@tok@na\endcsname{\def\PY@tc##1{\textcolor[rgb]{0.49,0.56,0.16}{##1}}}
\expandafter\def\csname PY@tok@nb\endcsname{\def\PY@tc##1{\textcolor[rgb]{0.00,0.50,0.00}{##1}}}
\expandafter\def\csname PY@tok@nc\endcsname{\let\PY@bf=\textbf\def\PY@tc##1{\textcolor[rgb]{0.00,0.00,1.00}{##1}}}
\expandafter\def\csname PY@tok@nd\endcsname{\def\PY@tc##1{\textcolor[rgb]{0.67,0.13,1.00}{##1}}}
\expandafter\def\csname PY@tok@ne\endcsname{\let\PY@bf=\textbf\def\PY@tc##1{\textcolor[rgb]{0.82,0.25,0.23}{##1}}}
\expandafter\def\csname PY@tok@nf\endcsname{\def\PY@tc##1{\textcolor[rgb]{0.00,0.00,1.00}{##1}}}
\expandafter\def\csname PY@tok@si\endcsname{\let\PY@bf=\textbf\def\PY@tc##1{\textcolor[rgb]{0.73,0.40,0.53}{##1}}}
\expandafter\def\csname PY@tok@s2\endcsname{\def\PY@tc##1{\textcolor[rgb]{0.73,0.13,0.13}{##1}}}
\expandafter\def\csname PY@tok@vi\endcsname{\def\PY@tc##1{\textcolor[rgb]{0.10,0.09,0.49}{##1}}}
\expandafter\def\csname PY@tok@nt\endcsname{\let\PY@bf=\textbf\def\PY@tc##1{\textcolor[rgb]{0.00,0.50,0.00}{##1}}}
\expandafter\def\csname PY@tok@nv\endcsname{\def\PY@tc##1{\textcolor[rgb]{0.10,0.09,0.49}{##1}}}
\expandafter\def\csname PY@tok@s1\endcsname{\def\PY@tc##1{\textcolor[rgb]{0.73,0.13,0.13}{##1}}}
\expandafter\def\csname PY@tok@sh\endcsname{\def\PY@tc##1{\textcolor[rgb]{0.73,0.13,0.13}{##1}}}
\expandafter\def\csname PY@tok@sc\endcsname{\def\PY@tc##1{\textcolor[rgb]{0.73,0.13,0.13}{##1}}}
\expandafter\def\csname PY@tok@sx\endcsname{\def\PY@tc##1{\textcolor[rgb]{0.00,0.50,0.00}{##1}}}
\expandafter\def\csname PY@tok@bp\endcsname{\def\PY@tc##1{\textcolor[rgb]{0.00,0.50,0.00}{##1}}}
\expandafter\def\csname PY@tok@c1\endcsname{\let\PY@it=\textit\def\PY@tc##1{\textcolor[rgb]{0.25,0.50,0.50}{##1}}}
\expandafter\def\csname PY@tok@kc\endcsname{\let\PY@bf=\textbf\def\PY@tc##1{\textcolor[rgb]{0.00,0.50,0.00}{##1}}}
\expandafter\def\csname PY@tok@c\endcsname{\let\PY@it=\textit\def\PY@tc##1{\textcolor[rgb]{0.25,0.50,0.50}{##1}}}
\expandafter\def\csname PY@tok@mf\endcsname{\def\PY@tc##1{\textcolor[rgb]{0.40,0.40,0.40}{##1}}}
\expandafter\def\csname PY@tok@err\endcsname{\def\PY@bc##1{\setlength{\fboxsep}{0pt}\fcolorbox[rgb]{1.00,0.00,0.00}{1,1,1}{\strut ##1}}}
\expandafter\def\csname PY@tok@kd\endcsname{\let\PY@bf=\textbf\def\PY@tc##1{\textcolor[rgb]{0.00,0.50,0.00}{##1}}}
\expandafter\def\csname PY@tok@ss\endcsname{\def\PY@tc##1{\textcolor[rgb]{0.10,0.09,0.49}{##1}}}
\expandafter\def\csname PY@tok@sr\endcsname{\def\PY@tc##1{\textcolor[rgb]{0.73,0.40,0.53}{##1}}}
\expandafter\def\csname PY@tok@mo\endcsname{\def\PY@tc##1{\textcolor[rgb]{0.40,0.40,0.40}{##1}}}
\expandafter\def\csname PY@tok@kn\endcsname{\let\PY@bf=\textbf\def\PY@tc##1{\textcolor[rgb]{0.00,0.50,0.00}{##1}}}
\expandafter\def\csname PY@tok@mi\endcsname{\def\PY@tc##1{\textcolor[rgb]{0.40,0.40,0.40}{##1}}}
\expandafter\def\csname PY@tok@gp\endcsname{\let\PY@bf=\textbf\def\PY@tc##1{\textcolor[rgb]{0.00,0.00,0.50}{##1}}}
\expandafter\def\csname PY@tok@o\endcsname{\def\PY@tc##1{\textcolor[rgb]{0.40,0.40,0.40}{##1}}}
\expandafter\def\csname PY@tok@kr\endcsname{\let\PY@bf=\textbf\def\PY@tc##1{\textcolor[rgb]{0.00,0.50,0.00}{##1}}}
\expandafter\def\csname PY@tok@s\endcsname{\def\PY@tc##1{\textcolor[rgb]{0.73,0.13,0.13}{##1}}}
\expandafter\def\csname PY@tok@kp\endcsname{\def\PY@tc##1{\textcolor[rgb]{0.00,0.50,0.00}{##1}}}
\expandafter\def\csname PY@tok@w\endcsname{\def\PY@tc##1{\textcolor[rgb]{0.73,0.73,0.73}{##1}}}
\expandafter\def\csname PY@tok@kt\endcsname{\def\PY@tc##1{\textcolor[rgb]{0.69,0.00,0.25}{##1}}}
\expandafter\def\csname PY@tok@ow\endcsname{\let\PY@bf=\textbf\def\PY@tc##1{\textcolor[rgb]{0.67,0.13,1.00}{##1}}}
\expandafter\def\csname PY@tok@sb\endcsname{\def\PY@tc##1{\textcolor[rgb]{0.73,0.13,0.13}{##1}}}
\expandafter\def\csname PY@tok@k\endcsname{\let\PY@bf=\textbf\def\PY@tc##1{\textcolor[rgb]{0.00,0.50,0.00}{##1}}}
\expandafter\def\csname PY@tok@se\endcsname{\let\PY@bf=\textbf\def\PY@tc##1{\textcolor[rgb]{0.73,0.40,0.13}{##1}}}
\expandafter\def\csname PY@tok@sd\endcsname{\let\PY@it=\textit\def\PY@tc##1{\textcolor[rgb]{0.73,0.13,0.13}{##1}}}

\def\PYZbs{\char`\\}
\def\PYZus{\char`\_}
\def\PYZob{\char`\{}
\def\PYZcb{\char`\}}
\def\PYZca{\char`\^}
\def\PYZam{\char`\&}
\def\PYZlt{\char`\<}
\def\PYZgt{\char`\>}
\def\PYZsh{\char`\#}
\def\PYZpc{\char`\%}
\def\PYZdl{\char`\$}
\def\PYZhy{\char`\-}
\def\PYZsq{\char`\'}
\def\PYZdq{\char`\"}
\def\PYZti{\char`\~}
% for compatibility with earlier versions
\def\PYZat{@}
\def\PYZlb{[}
\def\PYZrb{]}
\makeatother


    % Exact colors from NB
    \definecolor{incolor}{rgb}{0.0, 0.0, 0.5}
    \definecolor{outcolor}{rgb}{0.545, 0.0, 0.0}



    
    % Prevent overflowing lines due to hard-to-break entities
    \sloppy 
    % Setup hyperref package
    \hypersetup{
      breaklinks=true,  % so long urls are correctly broken across lines
      colorlinks=true,
      urlcolor=blue,
      linkcolor=darkorange,
      citecolor=darkgreen,
      }
    % Slightly bigger margins than the latex defaults
    
    \geometry{verbose,tmargin=1in,bmargin=1in,lmargin=1in,rmargin=1in}
    
    

    \begin{document}
    
    
    \maketitle
    
    

    
    \textbf{Francisco José Navarro-Brull}

    \emph{TÉCNICAS DE CÁLCULO NUMÉRICO APLICADAS A LA FÍSICA Y A LA QUÍMICA
- Máster en Ciencia de Materiales de la Universidad de Alicante}


    \section{OriginLab vs Python (comparación de gráficas)}


    El software \href{http://www.originlab.com/}{OriginLab®} es uno de lo
más utilizados en el mundo académico gracias a su versatilidad y
potencia de cálculo. En concreto, OriginLab® tiene una serie de ventajas
frente a sus \emph{competidores} (Excel®, MATLAB®\ldots{}) ya que logra
unir ambos mundos permitiendo a alguien acostumbrado a la interfaz del
primero realizar tareas de cálculo numérico más comunes del segundo.
Pero si hay algo que verdaderamente es útil para un investigador que
utiliza OriginLab®, son sus gráficas ``listas para publicar''.

En este contexto, \href{https://www.python.org/}{Python} cuenta con
magníficas librerías capaces de llevar a cabo tareas de cálculo numérico
(NumPy, SciPy) así para obtener gráficas (matplotlib) de calidad
equivalente o superior a OriginLab.

Si has leído con detenimiento hasta aquí y te suena el software citado
hasta ahora, la pregunta que te vendrá a la mente es ¿cómo un lenguaje
de programación como Python va a sustituir a OrginLab® y su interfaz
tipo Excel®?

Veamos, Python junto a sus librerías permite a día de hoy resolver el
ciclo de trabajo propio tal y como permite OriginLab®. Éstos son: 1.
Importar datos (xlrd, NumPy, csv, pandas) 2. Procesado de datos y
cálculo (SciPy, NumPy) 3. Visualización (matplotib) 4. Iteración pasos 2
y 3 (IPython Notebook, Spyder) 5. Publicación de resultados (matplotib)

    Entonces, \textbf{¿por qué no todo el mundo está usando Python? ¿cuál es
el problema?}

Bien, en todo este planteamiento Python \emph{falla} en un concepto.
OriginLab® no requiere conocimientos de programación para su uso más
básico, Python sí.

Por norma general:

\begin{itemize}
\itemsep1pt\parskip0pt\parsep0pt
\item
  Muchos científicos no tienen conocimientos de programación (o éstos
  son muy reducidos)
\item
  No tienen tiempo y quieren realizar estas tareas de la forma más
  rápida posible
\item
  Representar gráficas por comandos de texto puede ser frustrante
\end{itemize}

    Un fanático de Python (u otro lenguaje) te intentará convencer
hablándote de las ventajas de aprender a programar, que ciertamente son
muchas, pero esto te llevará tiempo (que \emph{no} tienes) y más
problemas (cuando tú lo que buscabas era una solución).


    \subsection{``Use the right tool''}


    Si te sientes cómodo utilizando OriginLab® y no le encuentras
limitaciones, sigue utilizándolo y amortiza los 850 o 1800 dólares que
vale su licencia para uso académico o profesional.

Si por el contrario, quieres (y dispones de tiempo para ello): *
Automatizar el importado y procesado de datos * Tener a tu alcance
algoritmos avanzados (y gratuitos) de estadística, machine learning,
inteligencia artificial, computer vision\ldots{} * Ahorrar en licencias
y poder usar este software en la empresa a la que vayas * Aprender a
programar en uno de los lenguajes más versátiles que existen * Hacer tu
investigación reproducible añadiendo a tu paper no sólo la información,
si no además las herramientas poder reutilizar tu trabajo y ganar más
impacto * Obtener gráficas para tus artículos de calidad igual o
superior a Origin

¡Eres bienvenido al mundo de Python!


    \subsection{¿De qué va este Notebook?}


    Como prueba de concepto, se utilizará IPython Notebook para llevar a
cabo un proceso de reproducción de muchas de las gráficas que OriginLab®
publicita. Puedes dar un vistazo a la
\href{http://www.originlab.com/www/products/graphgallery.aspx}{Galería
de OriginLab®} y \href{http://matplotlib.org/gallery.html}{Galería de
matplotib} para ver lo similares que pueden llegar a ser.

IPython Notebook un formato y solución elegante que ademas de código
puede contener texto, fórmulas mediante \LaTeX, vídeos, imágenes y
gráficas.

    Nota: La mejor manera de trabajar con matplotlib es buscar en su
\href{http://matplotlib.org/gallery.html}{galería} una gráfica similar
al resultado que queramos conseguir y utilizar dicho código como guía o
plantilla de trabajo


    \subsection{¿De qué \emph{NO} va este Notebook?}


    Pese a que se comentarán brevemente ciertas líneas de código, este
notebook no pretende ser un tutorial de matplotlib y/o Python. Para una
introducción a los mismos te recomendamos este
\href{http://cacheme.org/curso-online-python-cientifico-ingenieros/}{Curso
online (gratuito) de introducción a Python para científicos e
ingenieros} por {[}@Pybonacci{]}(http://pybonacci.wordpress.com) y
organizado por \href{http://cacheme.org}{CAChemE.org}.

Por cierto, también puedes echarle un vistazo a
\href{https://www.youtube.com/watch?v=_Bm8M9IwuFk}{Avoplot}, un proyecto
muy interesante que pretende simiplificar la vida a muchos científicos
pero que de momento se encuentra en fase de desarrollo.


    \subsection{Librerías gráficas en Python:}


    Los siguientes ejemplos harán uso de matplotlib y ajustarán el estilo de
las gráficas para hacerlas similares las de OriginLab®. Matplotlib puede
parecer viejo, estático y tener una configuración por defecto cutre. Sin
embargo, su uso es muy, muy extenso y ha demostrado ser una librería a
prueba de todo. Tal y como mencionaba
{[}@Pybonacci{]}(http://twitter.com/ptbonacci), es válido para el 99\%
de las personas. El 1\% restante tiene varias alternativas que elegir.
Muchas de ellas se basan matplotlib, lo que demuestra su robustez, como
\href{http://www.stanford.edu/~mwaskom/software/seaborn/}{seaborn},
\href{http://vincent.readthedocs.org/en/latest/}{vincent},
\href{http://blog.yhathq.com/posts/ggplot-for-python.html}{ggplot-py},
\href{http://olgabot.github.io/prettyplotlib/}{prettyplot},
\href{https://plot.ly/}{plot.ly},
\href{https://github.com/wrobstory/bearcart}{bearcart} o una de las más
recientes e interesantes \href{http://mpld3.github.io/}{mpld3}. Por otro
lado existen alternativas que han empezado un motor gráfico de Python
desde cero cómo por ejemplo \href{http://bokeh.pydata.org/}{Bokeh}.

En cualquier caso matplotlib otorga las siguientes funcionalidades.

\begin{itemize}
\itemsep1pt\parskip0pt\parsep0pt
\item
  API: Tipo MATLAB (por estados, tersa, menos poderosa) u orientada a
  objetos (sin estados, verbosa, más poderosa)
\item
  Abstracción: Básicamente un potente modelo interno de objetos SVG
  (vectoriales)
\item
  Gráficos de salida:

  \begin{itemize}
  \itemsep1pt\parskip0pt\parsep0pt
  \item
    Gráficos estáticos finales (backends) de los cuales puedas
    necesitar: pdf, png, svg, eps, ps, pgf, jpeg\ldots{}
  \item
    Así como interfaces gráficas (GUI backends): Tk, Agg, OSX, GTK, Qt4,
    WebAgg\ldots{}
  \end{itemize}
\end{itemize}

Para aprender más sobre matplotib es recomendable leer
\href{http://nbviewer.ipython.org/urls/raw.githubusercontent.com/jakevdp/OpenVisConf2014/master/Index.ipynb}{Python
in the browser age} por Jake VanderPlas (fuente original de lo descrito
en este último párrafo).


    \section{Comparación de capacidades gráficas (Origin vs Python)}


    En primer lugar, queremos que las gráficas aparezcan en este mismo
notebook por lo que damos la siguiente instrucción:

    \begin{Verbatim}[commandchars=\\\{\}]
{\color{incolor}In [{\color{incolor}1}]:} \PY{o}{\PYZpc{}}\PY{k}{matplotlib} \PY{n}{inline}
\end{Verbatim}

    Acto seguido importamos las librerías necesarias para la generación y
representación los resultados:

    \begin{Verbatim}[commandchars=\\\{\}]
{\color{incolor}In [{\color{incolor}2}]:} \PY{k+kn}{import} \PY{n+nn}{numpy} \PY{k+kn}{as} \PY{n+nn}{np}
        \PY{k+kn}{import} \PY{n+nn}{matplotlib.pyplot} \PY{k+kn}{as} \PY{n+nn}{plt}
\end{Verbatim}

    En caso de trabajar con Python 2.7:

    \begin{Verbatim}[commandchars=\\\{\}]
{\color{incolor}In [{\color{incolor}3}]:} \PY{k+kn}{from} \PY{n+nn}{\PYZus{}\PYZus{}future\PYZus{}\PYZus{}} \PY{k+kn}{import} \PY{n}{division}
\end{Verbatim}


    \subsubsection{Potencia de matplotlib: Resultados combinados con \LaTeX}


    Antes de empezar veamos un ejemplo de la capacidad de matplotlib para
generar figuras ``listas para publicar'' y su magnífica inegración con
\LaTeX. Además de representar valores numéricos, matplotlib permite la
añadir notas y texto en este formato. Un ejemplo de esta funcionalidad
es el siguiente:

    \begin{Verbatim}[commandchars=\\\{\}]
{\color{incolor}In [{\color{incolor}4}]:} \PY{c}{\PYZsh{} Los comentarios en Python se escriben empezando con una almohadilla}
        \PY{c}{\PYZsh{} estas líneas serán ignoradas en la ejecución }
        
        
        \PY{c}{\PYZsh{} Creamos la función que queremos representar (e integrar)}
        \PY{k}{def} \PY{n+nf}{func}\PY{p}{(}\PY{n}{x}\PY{p}{)}\PY{p}{:}
            \PY{k}{return} \PY{p}{(}\PY{n}{x} \PY{o}{\PYZhy{}} \PY{l+m+mi}{3}\PY{p}{)} \PY{o}{*} \PY{p}{(}\PY{n}{x} \PY{o}{\PYZhy{}} \PY{l+m+mi}{5}\PY{p}{)} \PY{o}{*} \PY{p}{(}\PY{n}{x} \PY{o}{\PYZhy{}} \PY{l+m+mi}{7}\PY{p}{)} \PY{o}{+} \PY{l+m+mi}{85}
        
        \PY{c}{\PYZsh{} Especificamos los límtes de integración}
        \PY{n}{a}\PY{p}{,} \PY{n}{b} \PY{o}{=} \PY{l+m+mi}{2}\PY{p}{,} \PY{l+m+mi}{9} 
        
        \PY{c}{\PYZsh{} Crea un vector con valores de 0 a 10}
        \PY{n}{x} \PY{o}{=} \PY{n}{np}\PY{o}{.}\PY{n}{linspace}\PY{p}{(}\PY{l+m+mi}{0}\PY{p}{,} \PY{l+m+mi}{10}\PY{p}{)}
        
        \PY{c}{\PYZsh{} Obtenemos los valores de y correspondientes}
        \PY{n}{y} \PY{o}{=} \PY{n}{func}\PY{p}{(}\PY{n}{x}\PY{p}{)}
        
        \PY{c}{\PYZsh{} Utilizamos matplotlib (este formato es orientado a clases)}
        \PY{c}{\PYZsh{} matplotlib también puede usarse tipo MATLAB(R)}
        \PY{n}{fig}\PY{p}{,} \PY{n}{ax} \PY{o}{=} \PY{n}{plt}\PY{o}{.}\PY{n}{subplots}\PY{p}{(}\PY{p}{)}
        
        \PY{c}{\PYZsh{} Dibuja con una línea roja de espesor determinado los resultados}
        \PY{n}{plt}\PY{o}{.}\PY{n}{plot}\PY{p}{(}\PY{n}{x}\PY{p}{,} \PY{n}{y}\PY{p}{,} \PY{l+s}{\PYZsq{}}\PY{l+s}{r}\PY{l+s}{\PYZsq{}}\PY{p}{,} \PY{n}{linewidth}\PY{o}{=}\PY{l+m+mi}{2}\PY{p}{)}
        
        \PY{c}{\PYZsh{} Establece el límite inferior del eje y}
        \PY{n}{plt}\PY{o}{.}\PY{n}{ylim}\PY{p}{(}\PY{n}{ymin}\PY{o}{=}\PY{l+m+mi}{0}\PY{p}{)}
        
        \PY{c}{\PYZsh{} Crea la región sombreada que representará la integral}
        \PY{n}{ix} \PY{o}{=} \PY{n}{np}\PY{o}{.}\PY{n}{linspace}\PY{p}{(}\PY{n}{a}\PY{p}{,} \PY{n}{b}\PY{p}{)}
        \PY{n}{iy} \PY{o}{=} \PY{n}{func}\PY{p}{(}\PY{n}{ix}\PY{p}{)}
        
        \PY{c}{\PYZsh{} Coordenadas de los puntos del polígono de la integral a representar}
        \PY{n}{verts} \PY{o}{=} \PY{p}{[}\PY{p}{(}\PY{n}{a}\PY{p}{,} \PY{l+m+mi}{0}\PY{p}{)}\PY{p}{]} \PY{o}{+} \PY{n+nb}{list}\PY{p}{(}\PY{n+nb}{zip}\PY{p}{(}\PY{n}{ix}\PY{p}{,} \PY{n}{iy}\PY{p}{)}\PY{p}{)} \PY{o}{+} \PY{p}{[}\PY{p}{(}\PY{n}{b}\PY{p}{,} \PY{l+m+mi}{0}\PY{p}{)}\PY{p}{]}
        
        \PY{c}{\PYZsh{} Carga la función Polygon de la librería}
        \PY{k+kn}{from} \PY{n+nn}{matplotlib.patches} \PY{k+kn}{import} \PY{n}{Polygon}
        
        \PY{c}{\PYZsh{} Representa dicho polígono cierto 90 y 50\PYZpc{} de transparencia para el relleno y borde.}
        \PY{n}{poly} \PY{o}{=} \PY{n}{Polygon}\PY{p}{(}\PY{n}{verts}\PY{p}{,} \PY{n}{facecolor}\PY{o}{=}\PY{l+s}{\PYZsq{}}\PY{l+s}{0.9}\PY{l+s}{\PYZsq{}}\PY{p}{,} \PY{n}{edgecolor}\PY{o}{=}\PY{l+s}{\PYZsq{}}\PY{l+s}{0.5}\PY{l+s}{\PYZsq{}}\PY{p}{)}
        \PY{n}{ax}\PY{o}{.}\PY{n}{add\PYZus{}patch}\PY{p}{(}\PY{n}{poly}\PY{p}{)}
        
        \PY{c}{\PYZsh{} Añade texto en latex con la siguiente instruccion}
        \PY{c}{\PYZsh{} plt.text(coordenada\PYZus{}x, coordenada\PYZus{}y, texto, opciones)}
        \PY{n}{plt}\PY{o}{.}\PY{n}{text}\PY{p}{(}\PY{l+m+mf}{0.5} \PY{o}{*} \PY{p}{(}\PY{n}{a} \PY{o}{+} \PY{n}{b}\PY{p}{)}\PY{p}{,} \PY{l+m+mi}{30}\PY{p}{,} \PY{l+s}{r\PYZdq{}}\PY{l+s}{\PYZdl{}}\PY{l+s}{\PYZbs{}}\PY{l+s}{int\PYZus{}a\PYZca{}b f(x)}\PY{l+s}{\PYZbs{}}\PY{l+s}{mathrm\PYZob{}d\PYZcb{}x\PYZdl{}}\PY{l+s}{\PYZdq{}}\PY{p}{,}
                 \PY{n}{horizontalalignment}\PY{o}{=}\PY{l+s}{\PYZsq{}}\PY{l+s}{center}\PY{l+s}{\PYZsq{}}\PY{p}{,} \PY{n}{fontsize}\PY{o}{=}\PY{l+m+mi}{20}\PY{p}{)}
        
        \PY{c}{\PYZsh{} Añade texto con posición relativa para indicar los ejes}
        \PY{n}{plt}\PY{o}{.}\PY{n}{figtext}\PY{p}{(}\PY{l+m+mf}{0.9}\PY{p}{,} \PY{l+m+mf}{0.05}\PY{p}{,} \PY{l+s}{\PYZsq{}}\PY{l+s}{\PYZdl{}x\PYZdl{}}\PY{l+s}{\PYZsq{}}\PY{p}{)}
        \PY{n}{plt}\PY{o}{.}\PY{n}{figtext}\PY{p}{(}\PY{l+m+mf}{0.1}\PY{p}{,} \PY{l+m+mf}{0.9}\PY{p}{,} \PY{l+s}{\PYZsq{}}\PY{l+s}{\PYZdl{}y\PYZdl{}}\PY{l+s}{\PYZsq{}}\PY{p}{)}
        
        \PY{c}{\PYZsh{} Ocutlamos los bordes superior y derecho del cuadro de la gráfica}
        \PY{n}{ax}\PY{o}{.}\PY{n}{spines}\PY{p}{[}\PY{l+s}{\PYZsq{}}\PY{l+s}{right}\PY{l+s}{\PYZsq{}}\PY{p}{]}\PY{o}{.}\PY{n}{set\PYZus{}visible}\PY{p}{(}\PY{n+nb+bp}{False}\PY{p}{)}
        \PY{n}{ax}\PY{o}{.}\PY{n}{spines}\PY{p}{[}\PY{l+s}{\PYZsq{}}\PY{l+s}{top}\PY{l+s}{\PYZsq{}}\PY{p}{]}\PY{o}{.}\PY{n}{set\PYZus{}visible}\PY{p}{(}\PY{n+nb+bp}{False}\PY{p}{)}
        
        \PY{c}{\PYZsh{} Añade marcas en el eje x con los límites de integración}
        \PY{c}{\PYZsh{} Coordenadas}
        \PY{n}{ax}\PY{o}{.}\PY{n}{set\PYZus{}xticks}\PY{p}{(}\PY{p}{(}\PY{n}{a}\PY{p}{,} \PY{n}{b}\PY{p}{)}\PY{p}{)}
        \PY{c}{\PYZsh{} Texto}
        \PY{n}{ax}\PY{o}{.}\PY{n}{set\PYZus{}xticklabels}\PY{p}{(}\PY{p}{(}\PY{l+s}{\PYZsq{}}\PY{l+s}{\PYZdl{}a\PYZdl{}}\PY{l+s}{\PYZsq{}}\PY{p}{,} \PY{l+s}{\PYZsq{}}\PY{l+s}{\PYZdl{}b\PYZdl{}}\PY{l+s}{\PYZsq{}}\PY{p}{)}\PY{p}{)}
        
        \PY{c}{\PYZsh{} Establece marcas sólo en el eje x inferior}
        \PY{n}{ax}\PY{o}{.}\PY{n}{xaxis}\PY{o}{.}\PY{n}{set\PYZus{}ticks\PYZus{}position}\PY{p}{(}\PY{l+s}{\PYZsq{}}\PY{l+s}{bottom}\PY{l+s}{\PYZsq{}}\PY{p}{)}
        
        \PY{c}{\PYZsh{} Elimina marcas del eje y}
        \PY{n}{ax}\PY{o}{.}\PY{n}{set\PYZus{}yticks}\PY{p}{(}\PY{p}{[}\PY{p}{]}\PY{p}{)}
        
        \PY{c}{\PYZsh{} Muestra figura}
        \PY{n}{plt}\PY{o}{.}\PY{n}{show}\PY{p}{(}\PY{p}{)}
        
        \PY{c}{\PYZsh{} Guarda figura en png o pdf}
        \PY{c}{\PYZsh{} Más formatosy y/o opciones mirar en la documentación}
        \PY{c}{\PYZsh{} http://matplotlib.org/faq/howto\PYZus{}faq.html\PYZsh{}save\PYZhy{}multiple\PYZhy{}plots\PYZhy{}to\PYZhy{}one\PYZhy{}pdf\PYZhy{}file}
        
        \PY{c}{\PYZsh{} Dado que llamamos a la figura \PYZdq{}fig\PYZdq{} cuando la creamos,}
        \PY{c}{\PYZsh{} llamamos al método .savefig}
        
        \PY{c}{\PYZsh{}fig.savefig(r\PYZsq{}figuras/mi\PYZus{}figura.pdf\PYZsq{})}
        \PY{c}{\PYZsh{}fig.savefig(r\PYZsq{}figuras/mi\PYZus{}figura.png\PYZsq{})}
\end{Verbatim}

    \begin{center}
    \adjustimage{max size={0.9\linewidth}{0.9\paperheight}}{originlab-python_files/originlab-python_24_0.png}
    \end{center}
    { \hspace*{\fill} \\}
    

    \subsubsection{Representación de errores}


    A pesar de que Excel® permite representar errores en los gráficos, esta
es una funcionalidad importante. Veamos como podemos hacer esto con
matplotib.

    \begin{Verbatim}[commandchars=\\\{\}]
{\color{incolor}In [{\color{incolor}5}]:} \PY{n}{n\PYZus{}grupos} \PY{o}{=} \PY{l+m+mi}{11}
        
        \PY{n}{valores} \PY{o}{=} \PY{p}{(}\PY{l+m+mf}{2.5}\PY{p}{,} \PY{l+m+mf}{7.5}\PY{p}{,} \PY{l+m+mi}{20}\PY{p}{,} \PY{l+m+mi}{26}\PY{p}{,} \PY{l+m+mi}{16}\PY{p}{,} \PY{l+m+mi}{11}\PY{p}{,} \PY{l+m+mf}{22.5}\PY{p}{,} \PY{l+m+mi}{24}\PY{p}{,} \PY{l+m+mi}{29}\PY{p}{,} \PY{l+m+mi}{25}\PY{p}{,} \PY{l+m+mi}{20}\PY{p}{)}
        \PY{n}{errores} \PY{o}{=} \PY{p}{(}\PY{l+m+mf}{0.5}\PY{p}{,} \PY{l+m+mi}{1}\PY{p}{,} \PY{l+m+mi}{2}\PY{p}{,} \PY{l+m+mi}{2}\PY{p}{,} \PY{l+m+mf}{1.5}\PY{p}{,} \PY{l+m+mi}{1}\PY{p}{,} \PY{l+m+mi}{2}\PY{p}{,} \PY{l+m+mi}{2}\PY{p}{,} \PY{l+m+mi}{2}\PY{p}{,} \PY{l+m+mi}{2}\PY{p}{,} \PY{l+m+mi}{2}\PY{p}{)}
        
        \PY{n}{fig}\PY{p}{,} \PY{n}{ax} \PY{o}{=} \PY{n}{plt}\PY{o}{.}\PY{n}{subplots}\PY{p}{(}\PY{p}{)}
        
        \PY{n}{ax}\PY{o}{.}\PY{n}{yaxis}\PY{o}{.}\PY{n}{grid}\PY{p}{(}\PY{p}{)}
        
        \PY{n}{index} \PY{o}{=} \PY{n}{np}\PY{o}{.}\PY{n}{arange}\PY{p}{(}\PY{n}{n\PYZus{}grupos}\PY{p}{)}
        \PY{n}{ancho\PYZus{}banda} \PY{o}{=} \PY{l+m+mf}{0.8}
        
        \PY{n}{opacity} \PY{o}{=} \PY{l+m+mf}{0.6}
        \PY{n}{error\PYZus{}config} \PY{o}{=} \PY{p}{\PYZob{}}\PY{l+s}{\PYZsq{}}\PY{l+s}{ecolor}\PY{l+s}{\PYZsq{}}\PY{p}{:} \PY{l+s}{\PYZsq{}}\PY{l+s}{0.3}\PY{l+s}{\PYZsq{}}\PY{p}{\PYZcb{}}
        
        \PY{n}{distancia\PYZus{}inicio} \PY{o}{=} \PY{l+m+mf}{0.5}
        
        \PY{n}{coordenadas\PYZus{}x\PYZus{}barras} \PY{o}{=} \PY{n}{index}\PY{o}{+}\PY{n}{distancia\PYZus{}inicio}
        
        \PY{n}{rects1} \PY{o}{=} \PY{n}{plt}\PY{o}{.}\PY{n}{bar}\PY{p}{(}\PY{n}{coordenadas\PYZus{}x\PYZus{}barras}\PY{p}{,} \PY{n}{valores}\PY{p}{,} \PY{n}{ancho\PYZus{}banda}\PY{p}{,}
                         \PY{n}{alpha}\PY{o}{=}\PY{n}{opacity}\PY{p}{,}
                         \PY{n}{color}\PY{o}{=}\PY{l+s}{\PYZsq{}}\PY{l+s}{orange}\PY{l+s}{\PYZsq{}}\PY{p}{,}
                         \PY{n}{yerr}\PY{o}{=}\PY{n}{errores}\PY{p}{,}
                         \PY{n}{error\PYZus{}kw}\PY{o}{=}\PY{n}{error\PYZus{}config}\PY{p}{,}
                         \PY{p}{)}
        
        \PY{n}{plt}\PY{o}{.}\PY{n}{xlabel}\PY{p}{(}\PY{l+s}{\PYZsq{}}\PY{l+s}{Bin}\PY{l+s}{\PYZsq{}}\PY{p}{)}
        \PY{n}{plt}\PY{o}{.}\PY{n}{ylabel}\PY{p}{(}\PY{l+s}{\PYZsq{}}\PY{l+s}{Count}\PY{l+s}{\PYZsq{}}\PY{p}{)}
        
        \PY{n}{etiquetas\PYZus{}eje\PYZus{}x} \PY{o}{=} \PY{p}{(}\PY{l+s}{\PYZsq{}}\PY{l+s}{7\PYZhy{}8}\PY{l+s}{\PYZsq{}}\PY{p}{,} \PY{l+s}{\PYZsq{}}\PY{l+s}{9\PYZhy{}10}\PY{l+s}{\PYZsq{}}\PY{p}{,} \PY{l+s}{\PYZsq{}}\PY{l+s}{11\PYZhy{}12}\PY{l+s}{\PYZsq{}}\PY{p}{,} \PY{l+s}{\PYZsq{}}\PY{l+s}{13\PYZhy{}14}\PY{l+s}{\PYZsq{}}\PY{p}{,} \PY{l+s}{\PYZsq{}}\PY{l+s}{15\PYZhy{}16}\PY{l+s}{\PYZsq{}}\PY{p}{,}
                           \PY{l+s}{\PYZsq{}}\PY{l+s}{17\PYZhy{}18}\PY{l+s}{\PYZsq{}}\PY{p}{,} \PY{l+s}{\PYZsq{}}\PY{l+s}{19\PYZhy{}20}\PY{l+s}{\PYZsq{}}\PY{p}{,} \PY{l+s}{\PYZsq{}}\PY{l+s}{21\PYZhy{}22}\PY{l+s}{\PYZsq{}}\PY{p}{,} \PY{l+s}{\PYZsq{}}\PY{l+s}{23\PYZhy{}24}\PY{l+s}{\PYZsq{}}\PY{p}{,} \PY{l+s}{\PYZsq{}}\PY{l+s}{25\PYZhy{}26}\PY{l+s}{\PYZsq{}}\PY{p}{,} \PY{l+s}{\PYZsq{}}\PY{l+s}{27\PYZhy{}28}\PY{l+s}{\PYZsq{}}\PY{p}{)}
        
        \PY{n}{coordenadas\PYZus{}etiquetas\PYZus{}x} \PY{o}{=} \PY{n}{index} \PY{o}{+} \PY{n}{ancho\PYZus{}banda}\PY{o}{/}\PY{l+m+mi}{2} \PY{o}{+} \PY{n}{distancia\PYZus{}inicio}
        
        \PY{n}{plt}\PY{o}{.}\PY{n}{xticks}\PY{p}{(}\PY{n}{coordenadas\PYZus{}etiquetas\PYZus{}x}\PY{p}{,} \PY{n}{etiquetas\PYZus{}eje\PYZus{}x}\PY{p}{)}
        \PY{n}{plt}\PY{o}{.}\PY{n}{tight\PYZus{}layout}\PY{p}{(}\PY{p}{)}
        
        \PY{c}{\PYZsh{} Opcional si queremos salvar las figuras como archivos}
        \PY{c}{\PYZsh{}fig.savefig(r\PYZsq{}figuras/barras\PYZus{}errores.pdf\PYZsq{})}
        \PY{c}{\PYZsh{}fig.savefig(r\PYZsq{}figuras/barras\PYZus{}errores.png\PYZsq{},dpi=150)}
        
        \PY{n}{plt}\PY{o}{.}\PY{n}{show}\PY{p}{(}\PY{p}{)}
\end{Verbatim}

    \begin{center}
    \adjustimage{max size={0.9\linewidth}{0.9\paperheight}}{originlab-python_files/originlab-python_27_0.png}
    \end{center}
    { \hspace*{\fill} \\}
    
    Resultado reproducido con Origin
\href{http://cloud.originlab.com/www/resources/graph_gallery/images_galleries_new/Error_Bar_Scatter_and_Column_Plot.gif}{(ver
online)}.


    \subsubsection{Diagramas de cajas (Whiskey)}


    Veamos algo más interesante, realizar
\href{http://es.wikipedia.org/wiki/Diagrama_de_caja}{diagramas de cajas
y bigotes} con Excel® ya no es tan sencillo. Sin embargo en matplotlib:

    \begin{Verbatim}[commandchars=\\\{\}]
{\color{incolor}In [{\color{incolor}6}]:} \PY{k}{def} \PY{n+nf}{fakeBootStrapper}\PY{p}{(}\PY{n}{n}\PY{p}{)}\PY{p}{:}
            \PY{l+s+sd}{\PYZsq{}\PYZsq{}\PYZsq{}}
        \PY{l+s+sd}{    Devuelve una mediana arbitraria e intervalos de confianza}
        \PY{l+s+sd}{    en una tupla}
        \PY{l+s+sd}{    \PYZsq{}\PYZsq{}\PYZsq{}}
            \PY{k}{if} \PY{n}{n} \PY{o}{==} \PY{l+m+mi}{1}\PY{p}{:}
                \PY{n}{med} \PY{o}{=} \PY{l+m+mf}{0.1}
                \PY{n}{CI} \PY{o}{=} \PY{p}{(}\PY{o}{\PYZhy{}}\PY{l+m+mf}{0.25}\PY{p}{,} \PY{l+m+mf}{0.25}\PY{p}{)}
            \PY{k}{else}\PY{p}{:}
                \PY{n}{med} \PY{o}{=} \PY{l+m+mf}{0.2}
                \PY{n}{CI} \PY{o}{=} \PY{p}{(}\PY{o}{\PYZhy{}}\PY{l+m+mf}{0.35}\PY{p}{,} \PY{l+m+mf}{0.50}\PY{p}{)}
        
            \PY{k}{return} \PY{n}{med}\PY{p}{,} \PY{n}{CI}
        
        
        \PY{c}{\PYZsh{} Fija una semilla para el random}
        \PY{n}{np}\PY{o}{.}\PY{n}{random}\PY{o}{.}\PY{n}{seed}\PY{p}{(}\PY{l+m+mi}{2}\PY{p}{)}
        \PY{n}{inc} \PY{o}{=} \PY{l+m+mf}{0.1}
        \PY{n}{e1} \PY{o}{=} \PY{n}{np}\PY{o}{.}\PY{n}{random}\PY{o}{.}\PY{n}{normal}\PY{p}{(}\PY{l+m+mi}{0}\PY{p}{,} \PY{l+m+mi}{1}\PY{p}{,} \PY{n}{size}\PY{o}{=}\PY{p}{(}\PY{l+m+mi}{500}\PY{p}{,}\PY{p}{)}\PY{p}{)}
        \PY{n}{e2} \PY{o}{=} \PY{n}{np}\PY{o}{.}\PY{n}{random}\PY{o}{.}\PY{n}{normal}\PY{p}{(}\PY{l+m+mi}{0}\PY{p}{,} \PY{l+m+mi}{1}\PY{p}{,} \PY{n}{size}\PY{o}{=}\PY{p}{(}\PY{l+m+mi}{500}\PY{p}{,}\PY{p}{)}\PY{p}{)}
        \PY{n}{e3} \PY{o}{=} \PY{n}{np}\PY{o}{.}\PY{n}{random}\PY{o}{.}\PY{n}{normal}\PY{p}{(}\PY{l+m+mi}{0}\PY{p}{,} \PY{l+m+mi}{1} \PY{o}{+} \PY{n}{inc}\PY{p}{,} \PY{n}{size}\PY{o}{=}\PY{p}{(}\PY{l+m+mi}{500}\PY{p}{,}\PY{p}{)}\PY{p}{)}
        \PY{n}{e4} \PY{o}{=} \PY{n}{np}\PY{o}{.}\PY{n}{random}\PY{o}{.}\PY{n}{normal}\PY{p}{(}\PY{l+m+mi}{0}\PY{p}{,} \PY{l+m+mi}{1} \PY{o}{+} \PY{l+m+mi}{2}\PY{o}{*}\PY{n}{inc}\PY{p}{,} \PY{n}{size}\PY{o}{=}\PY{p}{(}\PY{l+m+mi}{500}\PY{p}{,}\PY{p}{)}\PY{p}{)}
        
        \PY{n}{tratamientos} \PY{o}{=} \PY{p}{[}\PY{n}{e1}\PY{p}{,}\PY{n}{e2}\PY{p}{,}\PY{n}{e3}\PY{p}{,}\PY{n}{e4}\PY{p}{]}
        \PY{n}{med1}\PY{p}{,} \PY{n}{CI1} \PY{o}{=} \PY{n}{fakeBootStrapper}\PY{p}{(}\PY{l+m+mi}{1}\PY{p}{)}
        \PY{n}{med2}\PY{p}{,} \PY{n}{CI2} \PY{o}{=} \PY{n}{fakeBootStrapper}\PY{p}{(}\PY{l+m+mi}{2}\PY{p}{)}
        \PY{n}{medianas} \PY{o}{=} \PY{p}{[}\PY{n+nb+bp}{None}\PY{p}{,} \PY{n+nb+bp}{None}\PY{p}{,} \PY{n}{med1}\PY{p}{,} \PY{n}{med2}\PY{p}{]}
        \PY{n}{conf\PYZus{}intervalos} \PY{o}{=} \PY{p}{[}\PY{n+nb+bp}{None}\PY{p}{,} \PY{n+nb+bp}{None}\PY{p}{,} \PY{n}{CI1}\PY{p}{,} \PY{n}{CI2}\PY{p}{]}
        
        \PY{n}{fig}\PY{p}{,} \PY{n}{ax} \PY{o}{=} \PY{n}{plt}\PY{o}{.}\PY{n}{subplots}\PY{p}{(}\PY{p}{)}
        
        \PY{c}{\PYZsh{} Posiciones de las cajas}
        \PY{n}{pos} \PY{o}{=} \PY{n}{np}\PY{o}{.}\PY{n}{array}\PY{p}{(}\PY{n+nb}{range}\PY{p}{(}\PY{n+nb}{len}\PY{p}{(}\PY{n}{tratamientos}\PY{p}{)}\PY{p}{)}\PY{p}{)}\PY{o}{+}\PY{l+m+mi}{1}
        
        \PY{c}{\PYZsh{} Representa el diagrama}
        \PY{n}{bp} \PY{o}{=} \PY{n}{ax}\PY{o}{.}\PY{n}{boxplot}\PY{p}{(}\PY{n}{tratamientos}\PY{p}{,} \PY{n}{sym}\PY{o}{=}\PY{l+s}{\PYZsq{}}\PY{l+s}{k+}\PY{l+s}{\PYZsq{}}\PY{p}{,} \PY{n}{positions}\PY{o}{=}\PY{n}{pos}\PY{p}{,}
                        \PY{n}{notch}\PY{o}{=}\PY{l+m+mi}{1}\PY{p}{,} \PY{n}{bootstrap}\PY{o}{=}\PY{l+m+mi}{5000}\PY{p}{,}
                        \PY{n}{usermedians}\PY{o}{=}\PY{n}{medianas}\PY{p}{,}
                        \PY{n}{conf\PYZus{}intervals}\PY{o}{=}\PY{n}{conf\PYZus{}intervalos}\PY{p}{)}
        
        \PY{c}{\PYZsh{} Etiquetas para los ejes}
        \PY{n}{ax}\PY{o}{.}\PY{n}{set\PYZus{}xlabel}\PY{p}{(}\PY{l+s}{\PYZsq{}}\PY{l+s}{Tratamiento}\PY{l+s}{\PYZsq{}}\PY{p}{)}
        \PY{n}{ax}\PY{o}{.}\PY{n}{set\PYZus{}ylabel}\PY{p}{(}\PY{l+s}{\PYZsq{}}\PY{l+s}{Respuesta}\PY{l+s}{\PYZsq{}}\PY{p}{)}
        
        \PY{c}{\PYZsh{} Tipo de diagrama de cajas (Whisker)}
        \PY{n}{plt}\PY{o}{.}\PY{n}{setp}\PY{p}{(}\PY{n}{bp}\PY{p}{[}\PY{l+s}{\PYZsq{}}\PY{l+s}{whiskers}\PY{l+s}{\PYZsq{}}\PY{p}{]}\PY{p}{,} \PY{n}{color}\PY{o}{=}\PY{l+s}{\PYZsq{}}\PY{l+s}{k}\PY{l+s}{\PYZsq{}}\PY{p}{,}  \PY{n}{linestyle}\PY{o}{=}\PY{l+s}{\PYZsq{}}\PY{l+s}{\PYZhy{}}\PY{l+s}{\PYZsq{}} \PY{p}{)}
        \PY{n}{plt}\PY{o}{.}\PY{n}{setp}\PY{p}{(}\PY{n}{bp}\PY{p}{[}\PY{l+s}{\PYZsq{}}\PY{l+s}{fliers}\PY{l+s}{\PYZsq{}}\PY{p}{]}\PY{p}{,} \PY{n}{markersize}\PY{o}{=}\PY{l+m+mf}{5.0}\PY{p}{)}
        
        \PY{c}{\PYZsh{} Opcional si queremos salvar las figuras como archivos}
        \PY{c}{\PYZsh{}fig.savefig(r\PYZsq{}figuras/diagra\PYZhy{}cajas.pdf\PYZsq{})}
        \PY{c}{\PYZsh{}fig.savefig(r\PYZsq{}figuras/diagra\PYZhy{}cajas.png\PYZsq{},dpi=150)}
        
        \PY{n}{plt}\PY{o}{.}\PY{n}{show}\PY{p}{(}\PY{p}{)}
\end{Verbatim}

    \begin{center}
    \adjustimage{max size={0.9\linewidth}{0.9\paperheight}}{originlab-python_files/originlab-python_31_0.png}
    \end{center}
    { \hspace*{\fill} \\}
    
    Resultado reproducido con Origin
\href{http://cloud.originlab.com/www/resources/graph_gallery/images_galleries_new/Box_Width_by_Variable.png}{(ver
online)}.

    \subsubsection{Matriz de dispersión (Scatter
Matrix)}\label{matriz-de-dispersiuxf3n-scatter-matrix}

    \begin{Verbatim}[commandchars=\\\{\}]
{\color{incolor}In [{\color{incolor}30}]:} \PY{c}{\PYZsh{} La forma más sencilla de realizar esta visualización es}
         \PY{c}{\PYZsh{} mediante la librería pandas que permite trabajar de forma similar a R}
         
         \PY{k+kn}{import} \PY{n+nn}{pandas} \PY{k+kn}{as} \PY{n+nn}{pd}
         
         \PY{c}{\PYZsh{} Lee los datos (deben de estar en la carpeta de este Notebook)}
         \PY{c}{\PYZsh{} http://en.wikipedia.org/wiki/Iris\PYZus{}flower\PYZus{}data\PYZus{}set}
         
         \PY{n}{iris} \PY{o}{=} \PY{n}{pd}\PY{o}{.}\PY{n}{read\PYZus{}csv}\PY{p}{(}\PY{l+s}{\PYZdq{}}\PY{l+s}{data/iris.csv}\PY{l+s}{\PYZdq{}}\PY{p}{)}
         
         \PY{c}{\PYZsh{} Establece el dataframe}
         \PY{n}{df} \PY{o}{=} \PY{n}{pd}\PY{o}{.}\PY{n}{DataFrame}\PY{p}{(}\PY{n}{iris}\PY{p}{,} \PY{n}{columns}\PY{o}{=}\PY{p}{[}\PY{l+s}{\PYZsq{}}\PY{l+s}{Longitudo\PYZhy{}sepalo}\PY{l+s}{\PYZsq{}}\PY{p}{,} \PY{l+s}{\PYZsq{}}\PY{l+s}{Anchura\PYZhy{}sepalo}\PY{l+s}{\PYZsq{}}\PY{p}{,}
                                          \PY{l+s}{\PYZsq{}}\PY{l+s}{Longitud\PYZhy{}petalo}\PY{l+s}{\PYZsq{}}\PY{p}{,} \PY{l+s}{\PYZsq{}}\PY{l+s}{Anchura\PYZhy{}petalo}\PY{l+s}{\PYZsq{}}\PY{p}{]}\PY{p}{)}
         
         \PY{c}{\PYZsh{} Calcula y representa resultados}
         \PY{n}{pd}\PY{o}{.}\PY{n}{scatter\PYZus{}matrix}\PY{p}{(}\PY{n}{df}\PY{p}{,} \PY{n}{alpha}\PY{o}{=}\PY{l+m+mf}{0.6}\PY{p}{,} \PY{n}{figsize}\PY{o}{=}\PY{p}{(}\PY{l+m+mi}{10}\PY{p}{,}\PY{l+m+mi}{10}\PY{p}{)}\PY{p}{)}
         
         \PY{c}{\PYZsh{} Opcional si queremos salvar las figuras como archivos}
         \PY{c}{\PYZsh{}plt.savefig(r\PYZsq{}figuras/matriz\PYZhy{}dispersion.pdf\PYZsq{})}
         \PY{c}{\PYZsh{}plt.savefig(r\PYZsq{}figuras/matriz\PYZhy{}dispersion.png\PYZsq{},dpi=150)}
         
         \PY{n}{plt}\PY{o}{.}\PY{n}{show}\PY{p}{(}\PY{p}{)}
\end{Verbatim}

    \begin{center}
    \adjustimage{max size={0.9\linewidth}{0.9\paperheight}}{originlab-python_files/originlab-python_34_0.png}
    \end{center}
    { \hspace*{\fill} \\}
    
    Resultado reproducido con Origin
\href{http://cloud.originlab.com/www/resources/graph_gallery/images_galleries/Scatter_Matrix_1127.png}{(ver
online)}.


    \subsubsection{Diagramas con coordenadas polares:}


    \begin{Verbatim}[commandchars=\\\{\}]
{\color{incolor}In [{\color{incolor}28}]:} \PY{n}{N} \PY{o}{=} \PY{l+m+mi}{50}
         \PY{n}{radio} \PY{o}{=} \PY{l+m+mi}{2} \PY{o}{*} \PY{n}{np}\PY{o}{.}\PY{n}{random}\PY{o}{.}\PY{n}{rand}\PY{p}{(}\PY{n}{N}\PY{p}{)}
         \PY{n}{theta} \PY{o}{=} \PY{l+m+mi}{2} \PY{o}{*} \PY{n}{np}\PY{o}{.}\PY{n}{pi} \PY{o}{*} \PY{n}{np}\PY{o}{.}\PY{n}{random}\PY{o}{.}\PY{n}{rand}\PY{p}{(}\PY{n}{N}\PY{p}{)}
         \PY{n}{area} \PY{o}{=} \PY{l+m+mi}{75} \PY{o}{*} \PY{n}{radio}\PY{o}{*}\PY{o}{*}\PY{l+m+mi}{2} \PY{o}{*} \PY{n}{np}\PY{o}{.}\PY{n}{random}\PY{o}{.}\PY{n}{rand}\PY{p}{(}\PY{n}{N}\PY{p}{)}
         \PY{n}{colores} \PY{o}{=} \PY{n}{np}\PY{o}{.}\PY{n}{random}\PY{o}{.}\PY{n}{rand}\PY{p}{(}\PY{n}{N}\PY{p}{)}
         
         \PY{c}{\PYZsh{} Crea una figura con coordenadas polares}
         \PY{n}{ax} \PY{o}{=} \PY{n}{plt}\PY{o}{.}\PY{n}{subplot}\PY{p}{(}\PY{l+m+mi}{111}\PY{p}{,} \PY{n}{polar}\PY{o}{=}\PY{n+nb+bp}{True}\PY{p}{)}
         
         \PY{c}{\PYZsh{} Representa una diagrama de dispersión}
         \PY{n}{c} \PY{o}{=} \PY{n}{plt}\PY{o}{.}\PY{n}{scatter}\PY{p}{(}\PY{n}{theta}\PY{p}{,} \PY{n}{radio}\PY{p}{,} \PY{n}{c}\PY{o}{=}\PY{n}{colores}\PY{p}{,}
                         \PY{n}{s}\PY{o}{=}\PY{n}{area}\PY{p}{,} \PY{n}{alpha}\PY{o}{=}\PY{l+m+mf}{0.70}\PY{p}{)}
         
         \PY{c}{\PYZsh{} Opcional si queremos salvar las figuras como archivos}
         \PY{c}{\PYZsh{}plt.savefig(r\PYZsq{}figuras/coordenadas\PYZhy{}polares.pdf\PYZsq{})}
         \PY{c}{\PYZsh{}plt.savefig(r\PYZsq{}figuras/coordenadas\PYZhy{}polares.png\PYZsq{},dpi=150)}
         
         \PY{n}{plt}\PY{o}{.}\PY{n}{show}\PY{p}{(}\PY{p}{)}
\end{Verbatim}

    \begin{center}
    \adjustimage{max size={0.9\linewidth}{0.9\paperheight}}{originlab-python_files/originlab-python_37_0.png}
    \end{center}
    { \hspace*{\fill} \\}
    
    Resultado reproducido con Origin:


    \subsubsection{Diagramas de contorno:}


    Otro tipo de gráfico que Excel® presenta limtaciones son los diagramas
de contorno. Veamos cómo hacerlo con matplotlib:

    \begin{Verbatim}[commandchars=\\\{\}]
{\color{incolor}In [{\color{incolor}9}]:} \PY{c}{\PYZsh{} Función que queremos representar}
        \PY{k}{def} \PY{n+nf}{g}\PY{p}{(}\PY{n}{x}\PY{p}{,} \PY{n}{y}\PY{p}{)}\PY{p}{:}
            \PY{k}{return} \PY{o}{\PYZhy{}}\PY{p}{(}\PY{n}{np}\PY{o}{.}\PY{n}{cos}\PY{p}{(}\PY{n}{x}\PY{p}{)} \PY{o}{*} \PY{n}{np}\PY{o}{.}\PY{n}{sin}\PY{p}{(}\PY{n}{y}\PY{p}{)}\PY{p}{)}\PY{o}{*}\PY{o}{*}\PY{l+m+mi}{3}
        
        \PY{c}{\PYZsh{} Genera dos vectores para la malla}
        \PY{c}{\PYZsh{} valor numérico alto para ver resultados de la curva suaves}
        \PY{n}{x} \PY{o}{=} \PY{n}{np}\PY{o}{.}\PY{n}{linspace}\PY{p}{(}\PY{o}{\PYZhy{}}\PY{l+m+mi}{2}\PY{p}{,} \PY{l+m+mi}{4}\PY{p}{,} \PY{l+m+mi}{1000}\PY{p}{)}
        \PY{n}{y} \PY{o}{=} \PY{n}{np}\PY{o}{.}\PY{n}{linspace}\PY{p}{(}\PY{o}{\PYZhy{}}\PY{l+m+mi}{2}\PY{p}{,} \PY{l+m+mi}{3}\PY{p}{,} \PY{l+m+mi}{1000}\PY{p}{)}
        
        \PY{c}{\PYZsh{} Creamos la malla }
        \PY{n}{xx}\PY{p}{,} \PY{n}{yy} \PY{o}{=} \PY{n}{np}\PY{o}{.}\PY{n}{meshgrid}\PY{p}{(}\PY{n}{x}\PY{p}{,} \PY{n}{y}\PY{p}{)}
        
        \PY{c}{\PYZsh{} Calculamos los valores de la función para cada punto de la malla}
        \PY{n}{zz} \PY{o}{=} \PY{n}{g}\PY{p}{(}\PY{n}{xx}\PY{p}{,} \PY{n}{yy}\PY{p}{)}
        
        \PY{c}{\PYZsh{} Ajusta el tamaño de la figura con figsize}
        \PY{n}{fig}\PY{p}{,} \PY{n}{axes} \PY{o}{=} \PY{n}{plt}\PY{o}{.}\PY{n}{subplots}\PY{p}{(}\PY{n}{figsize}\PY{o}{=}\PY{p}{(}\PY{l+m+mi}{10}\PY{p}{,} \PY{l+m+mi}{8}\PY{p}{)}\PY{p}{)}
        
        \PY{c}{\PYZsh{} Asigna la salida a la variable cs para luego crear el colorbar}
        \PY{n}{cs} \PY{o}{=} \PY{n}{axes}\PY{o}{.}\PY{n}{contourf}\PY{p}{(}\PY{n}{xx}\PY{p}{,} \PY{n}{yy}\PY{p}{,} \PY{n}{zz}\PY{p}{,} \PY{n}{np}\PY{o}{.}\PY{n}{linspace}\PY{p}{(}\PY{o}{\PYZhy{}}\PY{l+m+mi}{1}\PY{p}{,} \PY{l+m+mi}{2}\PY{p}{,} \PY{l+m+mi}{100}\PY{p}{)}\PY{p}{,} \PY{n}{cmap}\PY{o}{=}\PY{n}{plt}\PY{o}{.}\PY{n}{cm}\PY{o}{.}\PY{n}{BrBG}\PY{p}{)}
        
        \PY{c}{\PYZsh{} Con `colors=\PYZsq{}k\PYZsq{}` dibujamos todas las líneas negras}
        \PY{c}{\PYZsh{} Asigna la salida a la variable cs2 para crear las etiquetas}
        \PY{n}{cs2} \PY{o}{=} \PY{n}{axes}\PY{o}{.}\PY{n}{contour}\PY{p}{(}\PY{n}{xx}\PY{p}{,} \PY{n}{yy}\PY{p}{,} \PY{n}{zz}\PY{p}{,} \PY{n}{np}\PY{o}{.}\PY{n}{linspace}\PY{p}{(}\PY{o}{\PYZhy{}}\PY{l+m+mi}{1}\PY{p}{,} \PY{l+m+mi}{2}\PY{p}{,} \PY{l+m+mi}{9}\PY{p}{)}\PY{p}{,} \PY{n}{colors}\PY{o}{=}\PY{l+s}{\PYZsq{}}\PY{l+s}{k}\PY{l+s}{\PYZsq{}}\PY{p}{)}
        
        \PY{c}{\PYZsh{} Crea las etiquetas sobre las líneas}
        \PY{n}{axes}\PY{o}{.}\PY{n}{clabel}\PY{p}{(}\PY{n}{cs2}\PY{p}{)}
        
        \PY{c}{\PYZsh{} Crea la barra de colores (nótese que pertenece a fig)}
        \PY{n}{fig}\PY{o}{.}\PY{n}{colorbar}\PY{p}{(}\PY{n}{cs}\PY{p}{)}
        
        \PY{c}{\PYZsh{} Ponemos las etiquetas de los ejes}
        \PY{n}{axes}\PY{o}{.}\PY{n}{set\PYZus{}xlabel}\PY{p}{(}\PY{l+s}{\PYZdq{}}\PY{l+s}{Eje x}\PY{l+s}{\PYZdq{}}\PY{p}{)}
        \PY{n}{axes}\PY{o}{.}\PY{n}{set\PYZus{}ylabel}\PY{p}{(}\PY{l+s}{\PYZdq{}}\PY{l+s}{Eje y}\PY{l+s}{\PYZdq{}}\PY{p}{)}
        \PY{n}{axes}\PY{o}{.}\PY{n}{set\PYZus{}title}\PY{p}{(}\PY{l+s}{u\PYZdq{}}\PY{l+s}{Función representada: \PYZdl{}g(x, y) = \PYZhy{} (}\PY{l+s}{\PYZbs{}}\PY{l+s}{cos\PYZob{}x\PYZcb{} }\PY{l+s}{\PYZbs{}}\PY{l+s}{, }\PY{l+s}{\PYZbs{}}\PY{l+s}{sin\PYZob{}y\PYZcb{})\PYZca{}3\PYZdl{}}\PY{l+s}{\PYZdq{}}\PY{p}{,}\PY{n}{fontsize}\PY{o}{=}\PY{l+m+mi}{20}\PY{p}{)}
        
        
        \PY{c}{\PYZsh{} Opcional si queremos salvar las figuras en archivos}
        \PY{c}{\PYZsh{}fig.savefig(r\PYZsq{}figuras/diagrama\PYZhy{}contorno.pdf\PYZsq{})}
        \PY{c}{\PYZsh{}fig.savefig(r\PYZsq{}figuras/diagrama\PYZhy{}contorno.png\PYZsq{},dpi=150)}
        
        \PY{c}{\PYZsh{} Muestra figura}
        \PY{n}{plt}\PY{o}{.}\PY{n}{show}\PY{p}{(}\PY{p}{)}
\end{Verbatim}

    \begin{center}
    \adjustimage{max size={0.9\linewidth}{0.9\paperheight}}{originlab-python_files/originlab-python_41_0.png}
    \end{center}
    { \hspace*{\fill} \\}
    
    Resultado reproducido con Origin
\href{http://cloud.originlab.com/www/resources/graph_gallery/images_galleries_new/Overlay_Contour_Plot.png}{(ver
online)}.


    \subsection{Gráficas en 3D}


    Una de las cosas más llamativas que publicita OriginLab® es el uso de su
herramienta para graficar resultados en 3D. Ciertamente, Excel® dispone
de pocas herramientas en ese aspecto y MATLAB® tiene el mismo problema
que Python (hay que aprender a programar) con el añadido del coste de
licencia extra. Por ello la solución más cómoda es usar OriginLab®. No
te ahorras la licencia pero te evitas tener que aprender programar.

    Existe una regla no escrita que dice algo así: \textbf{\emph{``si la
única forma de representar tus resultados es con una gráfica 3D, es que
algo estás haciendo mal''}}

Sin ánimo de ser tan radicales, vivimos en un mundo 2D. El papel o
pantalla donde se verán tus resultados son en 2D y por ello es mejor
evitar representar resultados mediante gráficas 3D. No obstante, para
exploración y visualización de resultados interactiva, son una
herramienta muy potente. Cabe destacar que matplotlib empezó a dar
soporte a figuras en 3D recientemente. Si se requieren visualizaciones
más complejas es recomendable usar
\href{http://docs.enthought.com/mayavi/mayavi/auto/examples.html}{Mayavi}

Veamos como hacer esto con matplotlib:

    \begin{Verbatim}[commandchars=\\\{\}]
{\color{incolor}In [{\color{incolor}10}]:} \PY{c}{\PYZsh{} Cargamos la librería 3D de matplotlib}
         \PY{k+kn}{from} \PY{n+nn}{mpl\PYZus{}toolkits.mplot3d} \PY{k+kn}{import} \PY{n}{axes3d}
         
         \PY{c}{\PYZsh{} Acceso rápido a los mapas de colores (colormap)}
         \PY{k+kn}{from} \PY{n+nn}{matplotlib} \PY{k+kn}{import} \PY{n}{cm}
\end{Verbatim}


    \subsection{Superficie en 3D}


    Empecemos con algo sencillo. Vamos a representar la superficie de un
parboloide.

    \begin{Verbatim}[commandchars=\\\{\}]
{\color{incolor}In [{\color{incolor}11}]:} \PY{c}{\PYZsh{} Crea los vectores x e y}
         \PY{n}{x} \PY{o}{=} \PY{n}{np}\PY{o}{.}\PY{n}{arange}\PY{p}{(}\PY{o}{\PYZhy{}}\PY{l+m+mi}{2}\PY{p}{,} \PY{l+m+mi}{2}\PY{p}{,} \PY{l+m+mf}{0.05}\PY{p}{)}
         \PY{n}{y} \PY{o}{=} \PY{n}{np}\PY{o}{.}\PY{n}{arange}\PY{p}{(}\PY{o}{\PYZhy{}}\PY{l+m+mi}{2}\PY{p}{,} \PY{l+m+mi}{2}\PY{p}{,} \PY{l+m+mf}{0.05}\PY{p}{)}
         
         \PY{c}{\PYZsh{} Genera el mallado 2D}
         \PY{n}{X}\PY{p}{,} \PY{n}{Y} \PY{o}{=} \PY{n}{np}\PY{o}{.}\PY{n}{meshgrid}\PY{p}{(}\PY{n}{x}\PY{p}{,}\PY{n}{y}\PY{p}{)}
         
         \PY{c}{\PYZsh{} Calcula Z: Paraboloide}
         \PY{n}{Z} \PY{o}{=} \PY{p}{(}\PY{n}{X}\PY{p}{)}\PY{o}{*}\PY{o}{*}\PY{l+m+mi}{2}\PY{o}{\PYZhy{}}\PY{p}{(}\PY{n}{Y}\PY{p}{)}\PY{o}{*}\PY{o}{*}\PY{l+m+mi}{2}
         
         \PY{c}{\PYZsh{} Tamaño de figura}
         \PY{n}{fig} \PY{o}{=} \PY{n}{plt}\PY{o}{.}\PY{n}{figure}\PY{p}{(}\PY{n}{figsize}\PY{o}{=}\PY{p}{(}\PY{l+m+mi}{8}\PY{p}{,}\PY{l+m+mi}{8}\PY{p}{)}\PY{p}{)}
         
         \PY{c}{\PYZsh{} Proyección de la figura en 3D}
         \PY{n}{ax} \PY{o}{=} \PY{n}{fig}\PY{o}{.}\PY{n}{gca}\PY{p}{(}\PY{n}{projection}\PY{o}{=}\PY{l+s}{\PYZsq{}}\PY{l+s}{3d}\PY{l+s}{\PYZsq{}}\PY{p}{)}
         
         \PY{c}{\PYZsh{} Dibuja los resultados}
         \PY{n}{ax}\PY{o}{.}\PY{n}{plot\PYZus{}surface}\PY{p}{(}\PY{n}{X}\PY{p}{,} \PY{n}{Y}\PY{p}{,} \PY{n}{Z}\PY{p}{,} \PY{n}{rstride}\PY{o}{=}\PY{l+m+mi}{1}\PY{p}{,} \PY{n}{cstride}\PY{o}{=}\PY{l+m+mi}{1}\PY{p}{,} \PY{n}{cmap}\PY{o}{=}\PY{n}{cm}\PY{o}{.}\PY{n}{jet}\PY{p}{,}
                 \PY{n}{linewidth}\PY{o}{=}\PY{l+m+mi}{0}\PY{p}{,} \PY{n}{antialiased}\PY{o}{=}\PY{n+nb+bp}{False}\PY{p}{)}
         
         \PY{c}{\PYZsh{} Texto de las etiquetas}
         \PY{n}{ax}\PY{o}{.}\PY{n}{set\PYZus{}xlabel}\PY{p}{(}\PY{l+s}{\PYZdq{}}\PY{l+s}{Eje X}\PY{l+s}{\PYZdq{}}\PY{p}{)}
         \PY{n}{ax}\PY{o}{.}\PY{n}{set\PYZus{}ylabel}\PY{p}{(}\PY{l+s}{\PYZdq{}}\PY{l+s}{Eje Y}\PY{l+s}{\PYZdq{}}\PY{p}{)}
         \PY{n}{ax}\PY{o}{.}\PY{n}{set\PYZus{}zlabel}\PY{p}{(}\PY{l+s}{\PYZdq{}}\PY{l+s}{Eje Z}\PY{l+s}{\PYZdq{}}\PY{p}{)}
         \PY{n}{ax}\PY{o}{.}\PY{n}{set\PYZus{}title}\PY{p}{(}\PY{l+s}{\PYZdq{}}\PY{l+s}{Paraboloide}\PY{l+s}{\PYZdq{}}\PY{p}{,} \PY{n}{fontsize}\PY{o}{=}\PY{l+m+mi}{16}\PY{p}{)}
         
         \PY{c}{\PYZsh{} Opcional si queremos salvar las figuras en archivos}
         \PY{c}{\PYZsh{}fig.savefig(r\PYZsq{}figuras/superficie\PYZhy{}3D.pdf\PYZsq{})}
         \PY{c}{\PYZsh{}fig.savefig(r\PYZsq{}figuras/superficie\PYZhy{}3D.png\PYZsq{},dpi=150)}
         
         \PY{n}{plt}\PY{o}{.}\PY{n}{show}\PY{p}{(}\PY{p}{)}
\end{Verbatim}

    \begin{center}
    \adjustimage{max size={0.9\linewidth}{0.9\paperheight}}{originlab-python_files/originlab-python_49_0.png}
    \end{center}
    { \hspace*{\fill} \\}
    
    Resultado reproducido con Origin
\href{http://cloud.originlab.com/www/resources/graph_gallery/images_galleries_new/3d_surface_from_virtual_matrix_opengl.png}{(ver
online)}.


    \subsubsection{Superficies 3D combinadas con diagramas de contorno}


    \begin{Verbatim}[commandchars=\\\{\}]
{\color{incolor}In [{\color{incolor}12}]:} \PY{n}{fig} \PY{o}{=} \PY{n}{plt}\PY{o}{.}\PY{n}{figure}\PY{p}{(}\PY{n}{figsize}\PY{o}{=}\PY{p}{(}\PY{l+m+mi}{8}\PY{p}{,}\PY{l+m+mi}{6}\PY{p}{)}\PY{p}{)}
         
         \PY{n}{ax} \PY{o}{=} \PY{n}{fig}\PY{o}{.}\PY{n}{gca}\PY{p}{(}\PY{n}{projection}\PY{o}{=}\PY{l+s}{\PYZsq{}}\PY{l+s}{3d}\PY{l+s}{\PYZsq{}}\PY{p}{)}
         
         \PY{c}{\PYZsh{} Carga datos de para test}
         \PY{n}{X}\PY{p}{,} \PY{n}{Y}\PY{p}{,} \PY{n}{Z} \PY{o}{=} \PY{n}{axes3d}\PY{o}{.}\PY{n}{get\PYZus{}test\PYZus{}data}\PY{p}{(}\PY{l+m+mf}{0.01}\PY{p}{)}
         
         \PY{n}{surf} \PY{o}{=} \PY{n}{ax}\PY{o}{.}\PY{n}{plot\PYZus{}surface}\PY{p}{(}\PY{n}{X}\PY{p}{,} \PY{n}{Y}\PY{p}{,} \PY{n}{Z}\PY{p}{,} \PY{n}{rstride}\PY{o}{=}\PY{l+m+mi}{10}\PY{p}{,} \PY{n}{cstride}\PY{o}{=}\PY{l+m+mi}{10}\PY{p}{,} \PY{n}{alpha}\PY{o}{=}\PY{l+m+mi}{1}\PY{p}{,}
                         \PY{n}{cmap}\PY{o}{=}\PY{n}{cm}\PY{o}{.}\PY{n}{jet}\PY{p}{,} \PY{n}{linewidth}\PY{o}{=}\PY{l+m+mf}{0.1}\PY{p}{)}
         
         \PY{c}{\PYZsh{} Proyección en la base}
         \PY{n}{cset} \PY{o}{=} \PY{n}{ax}\PY{o}{.}\PY{n}{contourf}\PY{p}{(}\PY{n}{X}\PY{p}{,} \PY{n}{Y}\PY{p}{,} \PY{n}{Z}\PY{p}{,} \PY{l+m+mi}{25}\PY{p}{,} \PY{n}{zdir}\PY{o}{=}\PY{l+s}{\PYZsq{}}\PY{l+s}{z}\PY{l+s}{\PYZsq{}}\PY{p}{,} \PY{n}{offset}\PY{o}{=}\PY{o}{\PYZhy{}}\PY{l+m+mi}{100}\PY{p}{,} \PY{n}{cmap}\PY{o}{=}\PY{n}{cm}\PY{o}{.}\PY{n}{jet}\PY{p}{,} \PY{p}{)}
         
         \PY{c}{\PYZsh{} Proyección de líneas en el lateral}
         \PY{n}{cset} \PY{o}{=} \PY{n}{ax}\PY{o}{.}\PY{n}{contour}\PY{p}{(}\PY{n}{X}\PY{p}{,} \PY{n}{Y}\PY{p}{,} \PY{n}{Z}\PY{p}{,} \PY{n}{zdir}\PY{o}{=}\PY{l+s}{\PYZsq{}}\PY{l+s}{y}\PY{l+s}{\PYZsq{}}\PY{p}{,} \PY{n}{offset}\PY{o}{=}\PY{o}{\PYZhy{}}\PY{l+m+mi}{30}\PY{p}{,} \PY{n}{cmap}\PY{o}{=}\PY{n}{cm}\PY{o}{.}\PY{n}{jet}\PY{p}{)}
         
         \PY{c}{\PYZsh{} Añade barra de color con valores de Z}
         \PY{n}{fig}\PY{o}{.}\PY{n}{colorbar}\PY{p}{(}\PY{n}{surf}\PY{p}{,} \PY{n}{shrink}\PY{o}{=}\PY{l+m+mf}{0.5}\PY{p}{,} \PY{n}{aspect}\PY{o}{=}\PY{l+m+mi}{10}\PY{p}{)}
         
         \PY{c}{\PYZsh{} Configuración de ejes y etiquetas}
         \PY{n}{ax}\PY{o}{.}\PY{n}{set\PYZus{}xlabel}\PY{p}{(}\PY{l+s}{\PYZsq{}}\PY{l+s}{X}\PY{l+s}{\PYZsq{}}\PY{p}{)}
         \PY{n}{ax}\PY{o}{.}\PY{n}{set\PYZus{}xlim}\PY{p}{(}\PY{o}{\PYZhy{}}\PY{l+m+mi}{30}\PY{p}{,} \PY{l+m+mi}{30}\PY{p}{)}
         \PY{n}{ax}\PY{o}{.}\PY{n}{set\PYZus{}ylabel}\PY{p}{(}\PY{l+s}{\PYZsq{}}\PY{l+s}{Y}\PY{l+s}{\PYZsq{}}\PY{p}{)}
         \PY{n}{ax}\PY{o}{.}\PY{n}{set\PYZus{}ylim}\PY{p}{(}\PY{o}{\PYZhy{}}\PY{l+m+mi}{30}\PY{p}{,} \PY{l+m+mi}{30}\PY{p}{)}
         \PY{n}{ax}\PY{o}{.}\PY{n}{set\PYZus{}zlabel}\PY{p}{(}\PY{l+s}{\PYZsq{}}\PY{l+s}{Z}\PY{l+s}{\PYZsq{}}\PY{p}{)}
         \PY{n}{ax}\PY{o}{.}\PY{n}{set\PYZus{}zlim}\PY{p}{(}\PY{o}{\PYZhy{}}\PY{l+m+mi}{100}\PY{p}{,} \PY{l+m+mi}{100}\PY{p}{)}
         
         \PY{n}{ax}\PY{o}{.}\PY{n}{set\PYZus{}title}\PY{p}{(}\PY{l+s}{\PYZdq{}}\PY{l+s}{Generado con matplotlib}\PY{l+s}{\PYZdq{}}\PY{p}{,} \PY{n}{fontsize}\PY{o}{=}\PY{l+m+mi}{16}\PY{p}{)}
         
         \PY{c}{\PYZsh{} Opcional si queremos salvar las figuras en archivos}
         \PY{c}{\PYZsh{}fig.savefig(r\PYZsq{}figuras/superficie\PYZhy{}3D\PYZhy{}combinado.pdf\PYZsq{})}
         \PY{c}{\PYZsh{}fig.savefig(r\PYZsq{}figuras/superficie\PYZhy{}3D\PYZhy{}combinado.png\PYZsq{},dpi=150)}
         
         \PY{n}{plt}\PY{o}{.}\PY{n}{show}\PY{p}{(}\PY{p}{)}
\end{Verbatim}

    \begin{center}
    \adjustimage{max size={0.9\linewidth}{0.9\paperheight}}{originlab-python_files/originlab-python_52_0.png}
    \end{center}
    { \hspace*{\fill} \\}
    
    Resultado reproducido con Origin
\href{http://cloud.originlab.com/www/resources/graph_gallery/images_galleries_new/3D_Surface_Plot.png}{(ver
online)}.

    \subsubsection{Cascada 3D}\label{cascada-3d}

    \begin{Verbatim}[commandchars=\\\{\}]
{\color{incolor}In [{\color{incolor}13}]:} \PY{k+kn}{from} \PY{n+nn}{mpl\PYZus{}toolkits.mplot3d} \PY{k+kn}{import} \PY{n}{Axes3D}
         \PY{k+kn}{from} \PY{n+nn}{matplotlib.collections} \PY{k+kn}{import} \PY{n}{PolyCollection}
         \PY{k+kn}{from} \PY{n+nn}{matplotlib.colors} \PY{k+kn}{import} \PY{n}{colorConverter}
         \PY{k+kn}{import} \PY{n+nn}{matplotlib.pyplot} \PY{k+kn}{as} \PY{n+nn}{plt}
         \PY{k+kn}{import} \PY{n+nn}{numpy} \PY{k+kn}{as} \PY{n+nn}{np}
         
         \PY{c}{\PYZsh{} Para poder crear las señales usaremos una funcion de Gauss}
         \PY{c}{\PYZsh{} y le añadiremos ruido blanco}
         
         \PY{k}{def} \PY{n+nf}{gauss\PYZus{}ruido}\PY{p}{(}\PY{n}{x}\PY{p}{,}\PY{n}{posicion}\PY{p}{,}\PY{n}{desviacion}\PY{p}{)}\PY{p}{:}
             \PY{l+s+sd}{\PYZsq{}\PYZsq{}\PYZsq{} Genera una curva de Gauss con ruido blanco}
         \PY{l+s+sd}{    con la posicion y desviacion estándar de entrada\PYZsq{}\PYZsq{}\PYZsq{}}
             \PY{n}{y} \PY{o}{=} \PY{n}{np}\PY{o}{.}\PY{n}{exp}\PY{p}{(} \PY{o}{\PYZhy{}}\PY{p}{(}\PY{n}{x}\PY{o}{\PYZhy{}}\PY{n}{posicion}\PY{p}{)}\PY{o}{*}\PY{o}{*}\PY{l+m+mi}{2}\PY{p}{)} \PY{o}{/} \PY{p}{(}\PY{l+m+mi}{2}\PY{o}{*}\PY{n}{np}\PY{o}{.}\PY{n}{sqrt}\PY{p}{(}\PY{n}{desviacion}\PY{p}{)}\PY{p}{)}
             \PY{n}{ruido} \PY{o}{=} \PY{l+m+mf}{0.01}\PY{o}{*} \PY{n}{np}\PY{o}{.}\PY{n}{random}\PY{o}{.}\PY{n}{randn}\PY{p}{(}\PY{n+nb}{len}\PY{p}{(}\PY{n}{x}\PY{p}{)}\PY{p}{)}
             \PY{n}{senyal} \PY{o}{=} \PY{n}{y} \PY{o}{+} \PY{n}{ruido}
             \PY{k}{return} \PY{n}{senyal}
         
         
         \PY{n}{fig} \PY{o}{=} \PY{n}{plt}\PY{o}{.}\PY{n}{figure}\PY{p}{(}\PY{n}{figsize}\PY{o}{=}\PY{p}{(}\PY{l+m+mi}{10}\PY{p}{,}\PY{l+m+mi}{6}\PY{p}{)}\PY{p}{)}
         \PY{n}{ax} \PY{o}{=} \PY{n}{fig}\PY{o}{.}\PY{n}{gca}\PY{p}{(}\PY{n}{projection}\PY{o}{=}\PY{l+s}{\PYZsq{}}\PY{l+s}{3d}\PY{l+s}{\PYZsq{}}\PY{p}{)}
         
         \PY{n}{cc} \PY{o}{=} \PY{k}{lambda} \PY{n}{arg}\PY{p}{:} \PY{n}{colorConverter}\PY{o}{.}\PY{n}{to\PYZus{}rgba}\PY{p}{(}\PY{n}{arg}\PY{p}{,} \PY{n}{alpha}\PY{o}{=}\PY{l+m+mf}{0.6}\PY{p}{)}
         
         
         
         \PY{n}{posicion}\PY{o}{=} \PY{p}{[}\PY{l+m+mi}{50}\PY{p}{,} \PY{l+m+mi}{60}\PY{p}{,} \PY{l+m+mi}{70}\PY{p}{,} \PY{l+m+mi}{80}\PY{p}{]}
         
         
         \PY{n}{verts} \PY{o}{=} \PY{p}{[}\PY{p}{]}
         \PY{k}{for} \PY{n}{pos} \PY{o+ow}{in} \PY{n}{posicion}\PY{p}{:}
             \PY{n}{x} \PY{o}{=} \PY{n}{np}\PY{o}{.}\PY{n}{linspace}\PY{p}{(}\PY{l+m+mi}{0}\PY{p}{,}\PY{l+m+mi}{150}\PY{p}{,}\PY{l+m+mi}{100}\PY{p}{)}
             
             \PY{n}{y} \PY{o}{=} \PY{n}{np}\PY{o}{.}\PY{n}{abs}\PY{p}{(}\PY{n}{gauss\PYZus{}ruido}\PY{p}{(}\PY{n}{x}\PY{p}{,}\PY{n}{pos}\PY{p}{,}\PY{l+m+mi}{5}\PY{p}{)}\PY{p}{)}
             
             \PY{n}{y}\PY{p}{[}\PY{l+m+mi}{0}\PY{p}{]}\PY{p}{,} \PY{n}{y}\PY{p}{[}\PY{o}{\PYZhy{}}\PY{l+m+mi}{1}\PY{p}{]} \PY{o}{=} \PY{l+m+mi}{0}\PY{p}{,} \PY{l+m+mi}{0}
             \PY{n}{verts}\PY{o}{.}\PY{n}{append}\PY{p}{(}\PY{n+nb}{list}\PY{p}{(}\PY{n+nb}{zip}\PY{p}{(}\PY{n}{x}\PY{p}{,} \PY{n}{y}\PY{p}{)}\PY{p}{)}\PY{p}{)}
         
         
         \PY{n}{poly} \PY{o}{=} \PY{n}{PolyCollection}\PY{p}{(}\PY{n}{verts}\PY{p}{,} \PY{n}{facecolors} \PY{o}{=} \PY{p}{[}\PY{n}{cc}\PY{p}{(}\PY{l+s}{\PYZsq{}}\PY{l+s}{r}\PY{l+s}{\PYZsq{}}\PY{p}{)}\PY{p}{,} \PY{n}{cc}\PY{p}{(}\PY{l+s}{\PYZsq{}}\PY{l+s}{g}\PY{l+s}{\PYZsq{}}\PY{p}{)}\PY{p}{,}
                                                    \PY{n}{cc}\PY{p}{(}\PY{l+s}{\PYZsq{}}\PY{l+s}{b}\PY{l+s}{\PYZsq{}}\PY{p}{)}\PY{p}{,} \PY{n}{cc}\PY{p}{(}\PY{l+s}{\PYZsq{}}\PY{l+s}{y}\PY{l+s}{\PYZsq{}}\PY{p}{)}\PY{p}{]}\PY{p}{)}
         \PY{n}{poly}\PY{o}{.}\PY{n}{set\PYZus{}alpha}\PY{p}{(}\PY{l+m+mf}{0.7}\PY{p}{)}
         
         \PY{c}{\PYZsh{} Posicion de z}
         \PY{n}{zs} \PY{o}{=} \PY{p}{[}\PY{l+m+mf}{0.1}\PY{p}{,} \PY{l+m+mf}{0.25}\PY{p}{,} \PY{l+m+mf}{0.50}\PY{p}{,} \PY{l+m+mf}{0.75}\PY{p}{]}
         
         \PY{c}{\PYZsh{} Dibuja los poligonos generados seleccionado}
         \PY{c}{\PYZsh{} la señal y como eje vertical}
         \PY{n}{ax}\PY{o}{.}\PY{n}{add\PYZus{}collection3d}\PY{p}{(}\PY{n}{poly}\PY{p}{,} \PY{n}{zs}\PY{o}{=}\PY{n}{zs}\PY{p}{,} \PY{n}{zdir}\PY{o}{=}\PY{l+s}{\PYZsq{}}\PY{l+s}{y}\PY{l+s}{\PYZsq{}}\PY{p}{)}
         
         \PY{n}{ax}\PY{o}{.}\PY{n}{set\PYZus{}xlabel}\PY{p}{(}\PY{l+s}{\PYZsq{}}\PY{l+s}{Espectro}\PY{l+s}{\PYZsq{}}\PY{p}{)}
         \PY{n}{ax}\PY{o}{.}\PY{n}{set\PYZus{}xlim3d}\PY{p}{(}\PY{l+m+mi}{0}\PY{p}{,} \PY{l+m+mi}{150}\PY{p}{)}
         \PY{n}{ax}\PY{o}{.}\PY{n}{set\PYZus{}ylabel}\PY{p}{(}\PY{l+s}{\PYZsq{}}\PY{l+s}{t muestra / s}\PY{l+s}{\PYZsq{}}\PY{p}{)}
         \PY{n}{ax}\PY{o}{.}\PY{n}{set\PYZus{}ylim3d}\PY{p}{(}\PY{l+m+mi}{0}\PY{p}{,} \PY{l+m+mi}{1}\PY{p}{)}
         \PY{n}{ax}\PY{o}{.}\PY{n}{set\PYZus{}zlabel}\PY{p}{(}\PY{l+s}{u\PYZsq{}}\PY{l+s}{C / mg/ml}\PY{l+s}{\PYZsq{}}\PY{p}{)}
         \PY{n}{ax}\PY{o}{.}\PY{n}{set\PYZus{}zlim3d}\PY{p}{(}\PY{l+m+mi}{0}\PY{p}{,} \PY{l+m+mf}{0.25}\PY{p}{)}
         
         \PY{c}{\PYZsh{} Ángulo}
         \PY{n}{ax}\PY{o}{.}\PY{n}{view\PYZus{}init}\PY{p}{(}\PY{l+m+mi}{20}\PY{p}{,}\PY{l+m+mi}{280}\PY{p}{)}
         
         \PY{c}{\PYZsh{} Opcional si queremos salvar las figuras en archivos}
         \PY{c}{\PYZsh{}fig.savefig(r\PYZsq{}figuras/cascada\PYZhy{}3D.pdf\PYZsq{})}
         \PY{c}{\PYZsh{}fig.savefig(r\PYZsq{}figuras/cascada\PYZhy{}3D.png\PYZsq{},dpi=150)}
         
         
         \PY{n}{plt}\PY{o}{.}\PY{n}{show}\PY{p}{(}\PY{p}{)}
\end{Verbatim}

    \begin{center}
    \adjustimage{max size={0.9\linewidth}{0.9\paperheight}}{originlab-python_files/originlab-python_55_0.png}
    \end{center}
    { \hspace*{\fill} \\}
    
    Resultado reproducido con Origin
\href{http://cloud.originlab.com/www/resources/graph_gallery/images_galleries/3D_waterfall_w_4plane_2.png}{(ver
online)}.


    \subsubsection{Diagramas de dispersión en 3D:}


    Por simplicidad, el siguiente ejemplo creará los datos con funciones de
números aleatorios. No obstante, se podría hacer el agrupamiento de
datos \href{http://en.wikipedia.org/wiki/Iris_flower_data_set}{Iris} con
Python y
\href{http://claudiovz.github.io/scipy-lecture-notes-ES/packages/scikit-learn/index.html\#k-means-clustering}{scikit-learn}.

    \begin{Verbatim}[commandchars=\\\{\}]
{\color{incolor}In [{\color{incolor}14}]:} \PY{n}{fig} \PY{o}{=} \PY{n}{plt}\PY{o}{.}\PY{n}{figure}\PY{p}{(}\PY{n}{figsize}\PY{o}{=}\PY{p}{(}\PY{l+m+mi}{10}\PY{p}{,}\PY{l+m+mi}{8}\PY{p}{)}\PY{p}{)}
         \PY{n}{ax} \PY{o}{=} \PY{n}{fig}\PY{o}{.}\PY{n}{gca}\PY{p}{(}\PY{n}{projection}\PY{o}{=}\PY{l+s}{\PYZsq{}}\PY{l+s}{3d}\PY{l+s}{\PYZsq{}}\PY{p}{)}
         
         \PY{c}{\PYZsh{} Genereamos los puntos con una distribución estándar normal de (n\PYZus{}puntos, dimensiones) }
         \PY{c}{\PYZsh{} y su repreesntación 3d junto a la proyección en el plano x\PYZhy{}y}
         \PY{n}{z\PYZus{}offset} \PY{o}{=} \PY{l+m+mi}{3}
         
         \PY{n}{coordenadas\PYZus{}clase1} \PY{o}{=} \PY{n}{np}\PY{o}{.}\PY{n}{array}\PY{p}{(}\PY{p}{[}\PY{l+m+mi}{2}\PY{p}{,}\PY{l+m+mi}{1}\PY{p}{,}\PY{l+m+mi}{5}\PY{p}{]}\PY{p}{)}
         \PY{n}{clase1} \PY{o}{=} \PY{l+m+mf}{0.15} \PY{o}{*} \PY{n}{np}\PY{o}{.}\PY{n}{random}\PY{o}{.}\PY{n}{standard\PYZus{}normal}\PY{p}{(}\PY{p}{(}\PY{l+m+mi}{50}\PY{p}{,}\PY{l+m+mi}{3}\PY{p}{)}\PY{p}{)} \PY{o}{+} \PY{n}{coordenadas\PYZus{}clase1}
         \PY{n}{ax}\PY{o}{.}\PY{n}{plot}\PY{p}{(}\PY{n}{clase1}\PY{p}{[}\PY{p}{:}\PY{p}{,}\PY{l+m+mi}{0}\PY{p}{]}\PY{p}{,}\PY{n}{clase1}\PY{p}{[}\PY{p}{:}\PY{p}{,}\PY{l+m+mi}{1}\PY{p}{]}\PY{p}{,}\PY{n}{clase1}\PY{p}{[}\PY{p}{:}\PY{p}{,}\PY{l+m+mi}{2}\PY{p}{]}\PY{p}{,}\PY{l+s}{\PYZsq{}}\PY{l+s}{ko}\PY{l+s}{\PYZsq{}}\PY{p}{,} \PY{n}{alpha}\PY{o}{=}\PY{l+m+mf}{0.6}\PY{p}{,} \PY{n}{label}\PY{o}{=}\PY{l+s}{\PYZsq{}}\PY{l+s}{Setosa}\PY{l+s}{\PYZsq{}}\PY{p}{)}
         
         \PY{n}{ax}\PY{o}{.}\PY{n}{plot}\PY{p}{(}\PY{n}{clase1}\PY{p}{[}\PY{p}{:}\PY{p}{,}\PY{l+m+mi}{0}\PY{p}{]}\PY{p}{,} \PY{n}{clase1}\PY{p}{[}\PY{p}{:}\PY{p}{,}\PY{l+m+mi}{1}\PY{p}{]}\PY{p}{,} \PY{n}{np}\PY{o}{.}\PY{n}{zeros\PYZus{}like}\PY{p}{(}\PY{n}{clase1}\PY{p}{[}\PY{p}{:}\PY{p}{,}\PY{l+m+mi}{2}\PY{p}{]}\PY{p}{)}\PY{o}{+}\PY{n}{z\PYZus{}offset}\PY{p}{,} \PY{l+s}{\PYZsq{}}\PY{l+s}{ko}\PY{l+s}{\PYZsq{}}\PY{p}{)}
         
         \PY{n}{coordenadas\PYZus{}clase2} \PY{o}{=} \PY{n}{np}\PY{o}{.}\PY{n}{array}\PY{p}{(}\PY{p}{[}\PY{l+m+mf}{3.5}\PY{p}{,}\PY{l+m+mf}{2.5}\PY{p}{,}\PY{l+m+mi}{6}\PY{p}{]}\PY{p}{)}
         \PY{n}{clase2} \PY{o}{=} \PY{l+m+mf}{0.3} \PY{o}{*} \PY{n}{np}\PY{o}{.}\PY{n}{random}\PY{o}{.}\PY{n}{standard\PYZus{}normal}\PY{p}{(}\PY{p}{(}\PY{l+m+mi}{50}\PY{p}{,}\PY{l+m+mi}{3}\PY{p}{)}\PY{p}{)} \PY{o}{+} \PY{n}{coordenadas\PYZus{}clase2}
         \PY{n}{ax}\PY{o}{.}\PY{n}{plot}\PY{p}{(}\PY{n}{clase2}\PY{p}{[}\PY{p}{:}\PY{p}{,}\PY{l+m+mi}{0}\PY{p}{]}\PY{p}{,}\PY{n}{clase2}\PY{p}{[}\PY{p}{:}\PY{p}{,}\PY{l+m+mi}{1}\PY{p}{]}\PY{p}{,}\PY{n}{clase2}\PY{p}{[}\PY{p}{:}\PY{p}{,}\PY{l+m+mi}{2}\PY{p}{]}\PY{p}{,}\PY{l+s}{\PYZsq{}}\PY{l+s}{ro}\PY{l+s}{\PYZsq{}}\PY{p}{,} \PY{n}{alpha}\PY{o}{=}\PY{l+m+mf}{0.6}\PY{p}{,} \PY{n}{label}\PY{o}{=}\PY{l+s}{\PYZsq{}}\PY{l+s}{Versicolor}\PY{l+s}{\PYZsq{}}\PY{p}{)}
         \PY{n}{ax}\PY{o}{.}\PY{n}{plot}\PY{p}{(}\PY{n}{clase2}\PY{p}{[}\PY{p}{:}\PY{p}{,}\PY{l+m+mi}{0}\PY{p}{]}\PY{p}{,} \PY{n}{clase2}\PY{p}{[}\PY{p}{:}\PY{p}{,}\PY{l+m+mi}{1}\PY{p}{]}\PY{p}{,} \PY{n}{np}\PY{o}{.}\PY{n}{zeros\PYZus{}like}\PY{p}{(}\PY{n}{clase2}\PY{p}{[}\PY{p}{:}\PY{p}{,}\PY{l+m+mi}{2}\PY{p}{]}\PY{p}{)}\PY{o}{+}\PY{n}{z\PYZus{}offset}\PY{p}{,} \PY{l+s}{\PYZsq{}}\PY{l+s}{ro}\PY{l+s}{\PYZsq{}}\PY{p}{)}
         
         \PY{n}{coordenadas\PYZus{}clase3} \PY{o}{=} \PY{n}{np}\PY{o}{.}\PY{n}{array}\PY{p}{(}\PY{p}{[}\PY{l+m+mi}{6}\PY{p}{,}\PY{l+m+mi}{3}\PY{p}{,}\PY{l+m+mi}{7}\PY{p}{]}\PY{p}{)}
         \PY{n}{clase3} \PY{o}{=} \PY{l+m+mf}{0.4} \PY{o}{*} \PY{n}{np}\PY{o}{.}\PY{n}{random}\PY{o}{.}\PY{n}{standard\PYZus{}normal}\PY{p}{(}\PY{p}{(}\PY{l+m+mi}{50}\PY{p}{,}\PY{l+m+mi}{3}\PY{p}{)}\PY{p}{)} \PY{o}{+} \PY{n}{coordenadas\PYZus{}clase3}
         \PY{n}{ax}\PY{o}{.}\PY{n}{plot}\PY{p}{(}\PY{n}{clase3}\PY{p}{[}\PY{p}{:}\PY{p}{,}\PY{l+m+mi}{0}\PY{p}{]}\PY{p}{,}\PY{n}{clase3}\PY{p}{[}\PY{p}{:}\PY{p}{,}\PY{l+m+mi}{1}\PY{p}{]}\PY{p}{,}\PY{n}{clase3}\PY{p}{[}\PY{p}{:}\PY{p}{,}\PY{l+m+mi}{2}\PY{p}{]}\PY{p}{,}\PY{l+s}{\PYZsq{}}\PY{l+s}{go}\PY{l+s}{\PYZsq{}}\PY{p}{,} \PY{n}{alpha}\PY{o}{=}\PY{l+m+mf}{0.6}\PY{p}{,} \PY{n}{label}\PY{o}{=}\PY{l+s}{\PYZsq{}}\PY{l+s}{Virginica}\PY{l+s}{\PYZsq{}}\PY{p}{)}
         \PY{n}{ax}\PY{o}{.}\PY{n}{plot}\PY{p}{(}\PY{n}{clase3}\PY{p}{[}\PY{p}{:}\PY{p}{,}\PY{l+m+mi}{0}\PY{p}{]}\PY{p}{,} \PY{n}{clase3}\PY{p}{[}\PY{p}{:}\PY{p}{,}\PY{l+m+mi}{1}\PY{p}{]}\PY{p}{,} \PY{n}{np}\PY{o}{.}\PY{n}{zeros\PYZus{}like}\PY{p}{(}\PY{n}{clase3}\PY{p}{[}\PY{p}{:}\PY{p}{,}\PY{l+m+mi}{2}\PY{p}{]}\PY{p}{)}\PY{o}{+}\PY{n}{z\PYZus{}offset}\PY{p}{,} \PY{l+s}{\PYZsq{}}\PY{l+s}{go}\PY{l+s}{\PYZsq{}}\PY{p}{)}
         
         \PY{c}{\PYZsh{} Generamos las esferas}
         
         \PY{n}{u1} \PY{o}{=} \PY{n}{np}\PY{o}{.}\PY{n}{linspace}\PY{p}{(}\PY{l+m+mi}{0}\PY{p}{,} \PY{l+m+mi}{2} \PY{o}{*} \PY{n}{np}\PY{o}{.}\PY{n}{pi}\PY{p}{,} \PY{l+m+mi}{100}\PY{p}{)}
         \PY{n}{v1} \PY{o}{=} \PY{n}{np}\PY{o}{.}\PY{n}{linspace}\PY{p}{(}\PY{l+m+mi}{0}\PY{p}{,} \PY{n}{np}\PY{o}{.}\PY{n}{pi}\PY{p}{,} \PY{l+m+mi}{100}\PY{p}{)}
         
         \PY{n}{x\PYZus{}esfera\PYZus{}1} \PY{o}{=} \PY{l+m+mi}{1}   \PY{o}{*} \PY{n}{np}\PY{o}{.}\PY{n}{outer}\PY{p}{(}\PY{n}{np}\PY{o}{.}\PY{n}{cos}\PY{p}{(}\PY{n}{u1}\PY{p}{)}\PY{p}{,} \PY{n}{np}\PY{o}{.}\PY{n}{sin}\PY{p}{(}\PY{n}{v1}\PY{p}{)}\PY{p}{)} \PY{o}{+} \PY{n}{coordenadas\PYZus{}clase1}\PY{p}{[}\PY{l+m+mi}{0}\PY{p}{]}
         \PY{n}{y\PYZus{}esfera\PYZus{}1} \PY{o}{=} \PY{l+m+mf}{0.5} \PY{o}{*} \PY{n}{np}\PY{o}{.}\PY{n}{outer}\PY{p}{(}\PY{n}{np}\PY{o}{.}\PY{n}{sin}\PY{p}{(}\PY{n}{u1}\PY{p}{)}\PY{p}{,} \PY{n}{np}\PY{o}{.}\PY{n}{sin}\PY{p}{(}\PY{n}{v1}\PY{p}{)}\PY{p}{)} \PY{o}{+} \PY{n}{coordenadas\PYZus{}clase1}\PY{p}{[}\PY{l+m+mi}{1}\PY{p}{]}
         \PY{n}{z\PYZus{}esfera\PYZus{}1} \PY{o}{=} \PY{l+m+mf}{1.5} \PY{o}{*} \PY{n}{np}\PY{o}{.}\PY{n}{outer}\PY{p}{(}\PY{n}{np}\PY{o}{.}\PY{n}{ones}\PY{p}{(}\PY{n}{np}\PY{o}{.}\PY{n}{size}\PY{p}{(}\PY{n}{u1}\PY{p}{)}\PY{p}{)}\PY{p}{,} \PY{n}{np}\PY{o}{.}\PY{n}{cos}\PY{p}{(}\PY{n}{v1}\PY{p}{)}\PY{p}{)} \PY{o}{+} \PY{n}{coordenadas\PYZus{}clase1}\PY{p}{[}\PY{l+m+mi}{2}\PY{p}{]}
         \PY{n}{ax}\PY{o}{.}\PY{n}{plot\PYZus{}surface}\PY{p}{(}\PY{n}{x\PYZus{}esfera\PYZus{}1}\PY{p}{,} \PY{n}{y\PYZus{}esfera\PYZus{}1}\PY{p}{,} \PY{n}{z\PYZus{}esfera\PYZus{}1}\PY{p}{,}
                         \PY{n}{rstride}\PY{o}{=}\PY{l+m+mi}{10}\PY{p}{,} \PY{n}{cstride}\PY{o}{=}\PY{l+m+mi}{10}\PY{p}{,} \PY{n}{linewidth}\PY{o}{=}\PY{l+m+mf}{0.1}\PY{p}{,} \PY{n}{color}\PY{o}{=}\PY{l+s}{\PYZsq{}}\PY{l+s}{b}\PY{l+s}{\PYZsq{}}\PY{p}{,} \PY{n}{alpha}\PY{o}{=}\PY{l+m+mf}{0.1}\PY{p}{)}
         
         \PY{n}{u2} \PY{o}{=} \PY{n}{np}\PY{o}{.}\PY{n}{linspace}\PY{p}{(}\PY{l+m+mi}{0}\PY{p}{,} \PY{l+m+mi}{2} \PY{o}{*} \PY{n}{np}\PY{o}{.}\PY{n}{pi}\PY{p}{,} \PY{l+m+mi}{100}\PY{p}{)}
         \PY{n}{v2} \PY{o}{=} \PY{n}{np}\PY{o}{.}\PY{n}{linspace}\PY{p}{(}\PY{l+m+mi}{0}\PY{p}{,} \PY{n}{np}\PY{o}{.}\PY{n}{pi}\PY{p}{,} \PY{l+m+mi}{100}\PY{p}{)}
         
         \PY{n}{x\PYZus{}esfera\PYZus{}2} \PY{o}{=} \PY{l+m+mf}{1.5} \PY{o}{*} \PY{n}{np}\PY{o}{.}\PY{n}{outer}\PY{p}{(}\PY{n}{np}\PY{o}{.}\PY{n}{cos}\PY{p}{(}\PY{n}{u2}\PY{p}{)}\PY{p}{,} \PY{n}{np}\PY{o}{.}\PY{n}{sin}\PY{p}{(}\PY{n}{v2}\PY{p}{)}\PY{p}{)} \PY{o}{+} \PY{n}{coordenadas\PYZus{}clase2}\PY{p}{[}\PY{l+m+mi}{0}\PY{p}{]}
         \PY{n}{y\PYZus{}esfera\PYZus{}2} \PY{o}{=} \PY{l+m+mi}{1}   \PY{o}{*} \PY{n}{np}\PY{o}{.}\PY{n}{outer}\PY{p}{(}\PY{n}{np}\PY{o}{.}\PY{n}{sin}\PY{p}{(}\PY{n}{u2}\PY{p}{)}\PY{p}{,} \PY{n}{np}\PY{o}{.}\PY{n}{sin}\PY{p}{(}\PY{n}{v2}\PY{p}{)}\PY{p}{)} \PY{o}{+} \PY{n}{coordenadas\PYZus{}clase2}\PY{p}{[}\PY{l+m+mi}{1}\PY{p}{]}
         \PY{n}{z\PYZus{}esfera\PYZus{}2} \PY{o}{=} \PY{l+m+mf}{1.8}   \PY{o}{*} \PY{n}{np}\PY{o}{.}\PY{n}{outer}\PY{p}{(}\PY{n}{np}\PY{o}{.}\PY{n}{ones}\PY{p}{(}\PY{n}{np}\PY{o}{.}\PY{n}{size}\PY{p}{(}\PY{n}{u2}\PY{p}{)}\PY{p}{)}\PY{p}{,} \PY{n}{np}\PY{o}{.}\PY{n}{cos}\PY{p}{(}\PY{n}{v2}\PY{p}{)}\PY{p}{)} \PY{o}{+} \PY{n}{coordenadas\PYZus{}clase2}\PY{p}{[}\PY{l+m+mi}{2}\PY{p}{]}
         \PY{n}{ax}\PY{o}{.}\PY{n}{plot\PYZus{}surface}\PY{p}{(}\PY{n}{x\PYZus{}esfera\PYZus{}2}\PY{p}{,} \PY{n}{y\PYZus{}esfera\PYZus{}2}\PY{p}{,} \PY{n}{z\PYZus{}esfera\PYZus{}2}\PY{p}{,}
                         \PY{n}{rstride}\PY{o}{=}\PY{l+m+mi}{10}\PY{p}{,} \PY{n}{cstride}\PY{o}{=}\PY{l+m+mi}{10}\PY{p}{,} \PY{n}{linewidth}\PY{o}{=}\PY{l+m+mf}{0.1}\PY{p}{,} \PY{n}{color}\PY{o}{=}\PY{l+s}{\PYZsq{}}\PY{l+s}{r}\PY{l+s}{\PYZsq{}}\PY{p}{,} \PY{n}{alpha}\PY{o}{=}\PY{l+m+mf}{0.1}\PY{p}{)}
         
         \PY{n}{u3} \PY{o}{=} \PY{n}{np}\PY{o}{.}\PY{n}{linspace}\PY{p}{(}\PY{l+m+mi}{0}\PY{p}{,} \PY{l+m+mi}{2} \PY{o}{*} \PY{n}{np}\PY{o}{.}\PY{n}{pi}\PY{p}{,} \PY{l+m+mi}{100}\PY{p}{)}
         \PY{n}{v3} \PY{o}{=} \PY{n}{np}\PY{o}{.}\PY{n}{linspace}\PY{p}{(}\PY{l+m+mi}{0}\PY{p}{,} \PY{n}{np}\PY{o}{.}\PY{n}{pi}\PY{p}{,} \PY{l+m+mi}{100}\PY{p}{)}
         
         \PY{n}{x\PYZus{}esfera\PYZus{}3} \PY{o}{=} \PY{l+m+mf}{1.5}   \PY{o}{*} \PY{n}{np}\PY{o}{.}\PY{n}{outer}\PY{p}{(}\PY{n}{np}\PY{o}{.}\PY{n}{cos}\PY{p}{(}\PY{n}{u3}\PY{p}{)}\PY{p}{,} \PY{n}{np}\PY{o}{.}\PY{n}{sin}\PY{p}{(}\PY{n}{v3}\PY{p}{)}\PY{p}{)} \PY{o}{+} \PY{n}{coordenadas\PYZus{}clase3}\PY{p}{[}\PY{l+m+mi}{0}\PY{p}{]}
         \PY{n}{y\PYZus{}esfera\PYZus{}3} \PY{o}{=} \PY{l+m+mi}{1}     \PY{o}{*} \PY{n}{np}\PY{o}{.}\PY{n}{outer}\PY{p}{(}\PY{n}{np}\PY{o}{.}\PY{n}{sin}\PY{p}{(}\PY{n}{u3}\PY{p}{)}\PY{p}{,} \PY{n}{np}\PY{o}{.}\PY{n}{sin}\PY{p}{(}\PY{n}{v3}\PY{p}{)}\PY{p}{)} \PY{o}{+} \PY{n}{coordenadas\PYZus{}clase3}\PY{p}{[}\PY{l+m+mi}{1}\PY{p}{]}
         \PY{n}{z\PYZus{}esfera\PYZus{}3} \PY{o}{=} \PY{l+m+mi}{2}   \PY{o}{*} \PY{n}{np}\PY{o}{.}\PY{n}{outer}\PY{p}{(}\PY{n}{np}\PY{o}{.}\PY{n}{ones}\PY{p}{(}\PY{n}{np}\PY{o}{.}\PY{n}{size}\PY{p}{(}\PY{n}{u3}\PY{p}{)}\PY{p}{)}\PY{p}{,} \PY{n}{np}\PY{o}{.}\PY{n}{cos}\PY{p}{(}\PY{n}{v3}\PY{p}{)}\PY{p}{)} \PY{o}{+} \PY{n}{coordenadas\PYZus{}clase3}\PY{p}{[}\PY{l+m+mi}{2}\PY{p}{]}
         \PY{n}{ax}\PY{o}{.}\PY{n}{plot\PYZus{}surface}\PY{p}{(}\PY{n}{x\PYZus{}esfera\PYZus{}3}\PY{p}{,} \PY{n}{y\PYZus{}esfera\PYZus{}3}\PY{p}{,} \PY{n}{z\PYZus{}esfera\PYZus{}3}\PY{p}{,}
                         \PY{n}{rstride}\PY{o}{=}\PY{l+m+mi}{10}\PY{p}{,} \PY{n}{cstride}\PY{o}{=}\PY{l+m+mi}{10}\PY{p}{,} \PY{n}{linewidth}\PY{o}{=}\PY{l+m+mf}{0.1}\PY{p}{,} \PY{n}{color}\PY{o}{=}\PY{l+s}{\PYZsq{}}\PY{l+s}{g}\PY{l+s}{\PYZsq{}}\PY{p}{,} \PY{n}{alpha}\PY{o}{=}\PY{l+m+mf}{0.1}\PY{p}{)}
         
         \PY{c}{\PYZsh{} Establcemos límites en los ejes y los etiquetamos}
         \PY{n}{ax}\PY{o}{.}\PY{n}{set\PYZus{}xlim3d}\PY{p}{(}\PY{l+m+mi}{0}\PY{p}{,} \PY{l+m+mi}{8}\PY{p}{)}
         \PY{n}{ax}\PY{o}{.}\PY{n}{set\PYZus{}ylim3d}\PY{p}{(}\PY{l+m+mi}{0}\PY{p}{,} \PY{l+m+mi}{4}\PY{p}{)}
         \PY{n}{ax}\PY{o}{.}\PY{n}{set\PYZus{}zlim3d}\PY{p}{(}\PY{n}{z\PYZus{}offset}\PY{p}{,} \PY{l+m+mi}{9}\PY{p}{)}
         
         \PY{n}{ax}\PY{o}{.}\PY{n}{set\PYZus{}xlabel}\PY{p}{(}\PY{l+s}{u\PYZsq{}}\PY{l+s}{Longitud del pétalo (cm)}\PY{l+s}{\PYZsq{}}\PY{p}{)}
         \PY{n}{ax}\PY{o}{.}\PY{n}{set\PYZus{}ylabel}\PY{p}{(}\PY{l+s}{u\PYZsq{}}\PY{l+s}{Anchura del pétalo (cm)}\PY{l+s}{\PYZsq{}}\PY{p}{)}
         \PY{n}{ax}\PY{o}{.}\PY{n}{set\PYZus{}zlabel}\PY{p}{(}\PY{l+s}{u\PYZsq{}}\PY{l+s}{Longitud del sépalo (cm)}\PY{l+s}{\PYZsq{}}\PY{p}{)}
         
         \PY{c}{\PYZsh{} Muestra leyenda}
         \PY{n}{ax}\PY{o}{.}\PY{n}{legend}\PY{p}{(}\PY{p}{)}
         
         \PY{c}{\PYZsh{} Opcional si queremos salvar las figuras en archivos}
         \PY{c}{\PYZsh{}fig.savefig(r\PYZsq{}figuras/dispersion\PYZhy{}3D.pdf\PYZsq{})}
         \PY{c}{\PYZsh{}fig.savefig(r\PYZsq{}figuras/dispersion\PYZhy{}3D.png\PYZsq{},dpi=150)}
         
         \PY{n}{plt}\PY{o}{.}\PY{n}{show}\PY{p}{(}\PY{p}{)}
\end{Verbatim}

    \begin{center}
    \adjustimage{max size={0.9\linewidth}{0.9\paperheight}}{originlab-python_files/originlab-python_59_0.png}
    \end{center}
    { \hspace*{\fill} \\}
    
    Resultado reproducido con Origin
\href{http://cloud.originlab.com/www/resources/graph_gallery/images_galleries_new/3D_Scatter_combined_with_Parametric_Surfaces.png}{(ver
online)}.


    \subsection{Un paso más allá}


    Como se ha comentado, Python dispone de diversas librerías. Una de las
más interesante es \href{http://sympy.org/en/index.html}{SymPy} que
proporciona poderosas funciones sistemas de álgebra computacional (CAS,
computer algebraic system). El siguiente código se puede ejecutar sin
necesidad de las anteriores celdas ya que es independiente. El propósito
es una demostración de la interactivdad con los nuevos
\href{http://nbviewer.ipython.org/github/ipython/ipython/blob/2.x/examples/Interactive\%20Widgets/Index.ipynb}{widgets
de IPython Notebook 2.0}.

    \begin{Verbatim}[commandchars=\\\{\}]
{\color{incolor}In [{\color{incolor}15}]:} \PY{c}{\PYZsh{} Cargamos widgets interactivos de IPython Notebook}
         
         \PY{k+kn}{from} \PY{n+nn}{IPython.html.widgets} \PY{k+kn}{import} \PY{n}{interact}
         \PY{k+kn}{from} \PY{n+nn}{IPython.display} \PY{k+kn}{import} \PY{n}{display}
\end{Verbatim}

    \begin{Verbatim}[commandchars=\\\{\}]
{\color{incolor}In [{\color{incolor}16}]:} \PY{c}{\PYZsh{} Cargamos las partes de SymPy que vamos a utilizar}
         
         \PY{k+kn}{from} \PY{n+nn}{sympy} \PY{k+kn}{import} \PY{n}{Symbol}\PY{p}{,} \PY{n}{Eq}\PY{p}{,} \PY{n}{factor}\PY{p}{,} \PY{n}{init\PYZus{}printing}
         \PY{n}{init\PYZus{}printing}\PY{p}{(}\PY{n}{use\PYZus{}latex}\PY{o}{=}\PY{l+s}{\PYZsq{}}\PY{l+s}{mathjax}\PY{l+s}{\PYZsq{}}\PY{p}{)}
\end{Verbatim}

    \begin{Verbatim}[commandchars=\\\{\}]
{\color{incolor}In [{\color{incolor}17}]:} \PY{c}{\PYZsh{} Creamos la variable simbólica x}
         \PY{n}{x} \PY{o}{=} \PY{n}{Symbol}\PY{p}{(}\PY{l+s}{\PYZsq{}}\PY{l+s}{x}\PY{l+s}{\PYZsq{}}\PY{p}{)}
\end{Verbatim}

    \begin{Verbatim}[commandchars=\\\{\}]
{\color{incolor}In [{\color{incolor}18}]:} \PY{c}{\PYZsh{} Creamos la función de factorización para el ejemplo}
         \PY{k}{def} \PY{n+nf}{factorit}\PY{p}{(}\PY{n}{n}\PY{p}{)}\PY{p}{:}
             \PY{n}{display}\PY{p}{(}\PY{n}{Eq}\PY{p}{(}\PY{n}{x}\PY{o}{*}\PY{o}{*}\PY{n}{n}\PY{o}{\PYZhy{}}\PY{l+m+mi}{1}\PY{p}{,} \PY{n}{factor}\PY{p}{(}\PY{n}{x}\PY{o}{*}\PY{o}{*}\PY{n}{n}\PY{o}{\PYZhy{}}\PY{l+m+mi}{1}\PY{p}{)}\PY{p}{)}\PY{p}{)}
\end{Verbatim}

    Cuando ejectuemos la función, nos devolverá un resultado en \LaTeX

    \begin{Verbatim}[commandchars=\\\{\}]
{\color{incolor}In [{\color{incolor}19}]:} \PY{n}{factorit}\PY{p}{(}\PY{l+m+mi}{12}\PY{p}{)}
\end{Verbatim}

    
        \begin{equation*}
        x^{12} - 1 = \left(x - 1\right) \left(x + 1\right) \left(x^{2} + 1\right) \left(x^{2} - x + 1\right) \left(x^{2} + x + 1\right) \left(x^{4} - x^{2} + 1\right)
        \end{equation*}

    
    \begin{Verbatim}[commandchars=\\\{\}]
{\color{incolor}In [{\color{incolor}20}]:} \PY{n}{interact}\PY{p}{(}\PY{n}{factorit}\PY{p}{,} \PY{n}{n}\PY{o}{=}\PY{p}{(}\PY{l+m+mi}{2}\PY{p}{,}\PY{l+m+mi}{20}\PY{p}{)}\PY{p}{)}\PY{p}{;}
\end{Verbatim}

    
        \begin{equation*}
        x^{11} - 1 = \left(x - 1\right) \left(x^{10} + x^{9} + x^{8} + x^{7} + x^{6} + x^{5} + x^{4} + x^{3} + x^{2} + x + 1\right)
        \end{equation*}

    
    Es hora de hacer algo más interesante. Hagamos una expansión por serie
de Taylor de $\frac{sin{x}}{x}$

    \begin{Verbatim}[commandchars=\\\{\}]
{\color{incolor}In [{\color{incolor}21}]:} \PY{k+kn}{from} \PY{n+nn}{sympy} \PY{k+kn}{import} \PY{n}{Symbol}\PY{p}{,} \PY{n}{sin}\PY{p}{,} \PY{n}{series}\PY{p}{,} \PY{n}{exp}
         \PY{n}{x} \PY{o}{=} \PY{n}{Symbol}\PY{p}{(}\PY{l+s}{\PYZsq{}}\PY{l+s}{x}\PY{l+s}{\PYZsq{}}\PY{p}{)}
         
         \PY{n}{ecuacion\PYZus{}ejemplo} \PY{o}{=}  \PY{n}{sin}\PY{p}{(}\PY{n}{x}\PY{p}{)}\PY{o}{/}\PY{n}{x}
         \PY{n}{orden} \PY{o}{=} \PY{l+m+mi}{5}
         
         \PY{n}{series}\PY{p}{(}\PY{n}{ecuacion\PYZus{}ejemplo}\PY{p}{,} \PY{n}{x}\PY{p}{,} \PY{n}{n}\PY{o}{=}\PY{n}{orden}\PY{p}{)}
\end{Verbatim}
\texttt{\color{outcolor}Out[{\color{outcolor}21}]:}
    
    
        \begin{equation*}
        1 - \frac{x^{2}}{6} + \frac{x^{4}}{120} + \mathcal{O}\left(x^{5}\right)
        \end{equation*}

    

    Vamos a representar los resultados de forma interactiva

    \begin{Verbatim}[commandchars=\\\{\}]
{\color{incolor}In [{\color{incolor}22}]:} \PY{k+kn}{from} \PY{n+nn}{sympy.plotting} \PY{k+kn}{import} \PY{n}{plot}
         
         \PY{c}{\PYZsh{} Representa las figuras en línea con el documento (no en una ventana emergente)}
         \PY{o}{\PYZpc{}}\PY{k}{matplotlib} \PY{n}{inline}
\end{Verbatim}

    \begin{Verbatim}[commandchars=\\\{\}]
{\color{incolor}In [{\color{incolor}23}]:} \PY{k}{def} \PY{n+nf}{taylor\PYZus{}graf}\PY{p}{(}\PY{n}{n}\PY{p}{)}\PY{p}{:}
             
             \PY{n}{e} \PY{o}{=} \PY{n}{exp}\PY{p}{(}\PY{l+m+mi}{1}\PY{p}{)}
             \PY{n}{ecuacion} \PY{o}{=} \PY{n}{e}\PY{o}{*}\PY{o}{*}\PY{n}{x}
             
             \PY{c}{\PYZsh{}calculamos la expansion elimnando el termino del error}
             \PY{n}{ecuacion\PYZus{}aprox} \PY{o}{=} \PY{n}{series}\PY{p}{(}\PY{n}{ecuacion}\PY{p}{,} \PY{n}{x}\PY{p}{,} \PY{n}{n}\PY{o}{=}\PY{n}{n}\PY{o}{+}\PY{l+m+mi}{1}\PY{p}{)}\PY{o}{.}\PY{n}{removeO}\PY{p}{(}\PY{p}{)}
             
             \PY{n}{p1} \PY{o}{=} \PY{n}{plot}\PY{p}{(}\PY{n}{ecuacion}\PY{p}{,} \PY{p}{(}\PY{n}{x}\PY{p}{,} \PY{o}{\PYZhy{}}\PY{l+m+mi}{3}\PY{p}{,} \PY{l+m+mi}{3}\PY{p}{)}\PY{p}{,} \PY{n}{show}\PY{o}{=}\PY{n+nb+bp}{False}\PY{p}{,} \PY{n}{line\PYZus{}color}\PY{o}{=}\PY{l+s}{\PYZsq{}}\PY{l+s}{b}\PY{l+s}{\PYZsq{}}\PY{p}{,} \PY{n}{label}\PY{o}{=}\PY{l+s}{\PYZsq{}}\PY{l+s}{ecuacion}\PY{l+s}{\PYZsq{}}\PY{p}{)}
             \PY{n}{p2} \PY{o}{=} \PY{n}{plot}\PY{p}{(}\PY{n}{ecuacion\PYZus{}aprox}\PY{p}{,} \PY{p}{(}\PY{n}{x}\PY{p}{,} \PY{o}{\PYZhy{}}\PY{l+m+mi}{3}\PY{p}{,} \PY{l+m+mi}{3}\PY{p}{)}\PY{p}{,} \PY{n}{show}\PY{o}{=}\PY{n+nb+bp}{False}\PY{p}{,} \PY{n}{line\PYZus{}color}\PY{o}{=}\PY{l+s}{\PYZsq{}}\PY{l+s}{r}\PY{l+s}{\PYZsq{}}\PY{p}{,} \PY{n}{label}\PY{o}{=}\PY{l+s}{\PYZsq{}}\PY{l+s}{aprox}\PY{l+s}{\PYZsq{}}\PY{p}{)}
             
             \PY{c}{\PYZsh{} Haz la segunda función parte de la primera}
             \PY{n}{p1}\PY{o}{.}\PY{n}{extend}\PY{p}{(}\PY{n}{p2}\PY{p}{)}
             \PY{c}{\PYZsh{} Nota: Este código se debe a que representar dos funciones juntas con diferentes colores }
             \PY{c}{\PYZsh{} aun no está implmentado}
             \PY{c}{\PYZsh{} http://stackoverflow.com/questions/21429866/change\PYZhy{}color\PYZhy{}implicit\PYZhy{}plot\PYZhy{}sympy}
             
             \PY{n}{p1}\PY{o}{.}\PY{n}{show}\PY{p}{(}\PY{p}{)}
\end{Verbatim}

    Función exponencial $e^x$ (en azul) y la suma de los primeros n+1
términos

    de su serie de Taylor en torno a cero (en rojo).

    \begin{Verbatim}[commandchars=\\\{\}]
{\color{incolor}In [{\color{incolor}24}]:} \PY{n}{interact}\PY{p}{(}\PY{n}{taylor\PYZus{}graf}\PY{p}{,} \PY{n}{n}\PY{o}{=}\PY{p}{(}\PY{l+m+mi}{0}\PY{p}{,}\PY{l+m+mi}{10}\PY{p}{)}\PY{p}{)}\PY{p}{;}
\end{Verbatim}

    \begin{center}
    \adjustimage{max size={0.9\linewidth}{0.9\paperheight}}{originlab-python_files/originlab-python_77_0.png}
    \end{center}
    { \hspace*{\fill} \\}
    
%    Si estás viendo este IPython Notebook offline, este es el resultado:
%\includegraphics{figuras/taylor-exp.gif}

    Por último, SymPy también representa expresiones matemáticas en 3D:

    \begin{Verbatim}[commandchars=\\\{\}]
{\color{incolor}In [{\color{incolor}25}]:} \PY{k+kn}{from} \PY{n+nn}{sympy} \PY{k+kn}{import} \PY{n}{symbols}
         \PY{k+kn}{from} \PY{n+nn}{sympy.plotting} \PY{k+kn}{import} \PY{n}{plot3d}
         
         \PY{n}{x}\PY{p}{,} \PY{n}{y} \PY{o}{=} \PY{n}{symbols}\PY{p}{(}\PY{l+s}{\PYZsq{}}\PY{l+s}{x y}\PY{l+s}{\PYZsq{}}\PY{p}{)}
\end{Verbatim}

    \begin{Verbatim}[commandchars=\\\{\}]
{\color{incolor}In [{\color{incolor}29}]:} \PY{n}{plot3d}\PY{p}{(}\PY{n+nb}{abs}\PY{p}{(}\PY{n}{x}\PY{o}{*}\PY{n}{y}\PY{p}{)}\PY{p}{,} \PY{p}{(}\PY{n}{x}\PY{p}{,} \PY{o}{\PYZhy{}}\PY{l+m+mi}{5}\PY{p}{,} \PY{l+m+mi}{5}\PY{p}{)}\PY{p}{,} \PY{p}{(}\PY{n}{y}\PY{p}{,} \PY{o}{\PYZhy{}}\PY{l+m+mi}{5}\PY{p}{,} \PY{l+m+mi}{5}\PY{p}{)}\PY{p}{)}
\end{Verbatim}

    \begin{center}
    \adjustimage{max size={0.9\linewidth}{0.9\paperheight}}{originlab-python_files/originlab-python_81_0.png}
    \end{center}
    { \hspace*{\fill} \\}
    
            \begin{Verbatim}[commandchars=\\\{\}]
{\color{outcolor}Out[{\color{outcolor}29}]:} <sympy.plotting.plot.Plot at 0x1370da58>
\end{Verbatim}
        
    Este IPython Notebook puede imprimirse en PDF generando (previamente) un
archivo ``.tex'' \LaTeX mediante
\href{http://johnmacfarlane.net/pandoc/}{pandoc}. Por ejemplo para
obtener el archivo basta con ejecutar en la línea de comandos el
siguiente código

\texttt{ipython nbconvert -{}-to latex originlab-python.ipynb}

Si se desea obtener más información, consultar
\href{http://ipython.org/ipython-doc/stable/notebook/nbconvert.html}{documentación}.


    % Add a bibliography block to the postdoc
    
    
    
    \end{document}
